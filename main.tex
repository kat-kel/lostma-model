\documentclass[a4paper]{article}

\usepackage[utf8]{inputenc}

% \usepackage{enumitem}
\usepackage{listings}
\usepackage{hyperref}
\usepackage{booktabs}
\usepackage{xurl}
\usepackage{multicol}
\usepackage{marginnote}

% color row
\usepackage[table, svgnames, dvipsnames]{xcolor}
\usepackage{makecell, cellspace, caption}

% \usepackage{geometry}
% \geometry{margin=1in}
\usepackage{multirow}
\usepackage{longtable}
\usepackage{er-diagram}
\usepackage{smartdiagram}
\usepackage{circuitikz}
\usetikzlibrary{calc,shapes.geometric,positioning,arrows,mindmap}
\pgfdeclarelayer{bg}    % declare background layer
\pgfsetlayers{bg,main}

\counterwithin{figure}{section}
\counterwithin{table}{section}


%Multiple authors
\usepackage{authblk}
%ENC biblio style
\usepackage[backend=biber, style=verbose%
% authoryear-comp, mergedate=compact,
, hyperref=true, uniquename=init, giveninits=true]{biblatex}
\addbibresource{bibliography.bib}

%\DeclareNameWrapperFormat{sortname}{\mkbibbold{#1}}
 
%for multiple graphics related
\usepackage{subcaption}

\usepackage{graphicx}
\usepackage{tikz}

\usepackage{csquotes}

\title{Modeling Medieval Romances}

\author{Kelly Christensen}
\affil{École nationale des chartes, 65 Rue de Richelieu, 75002 Paris, \texttt{kelly.christensen@chartes.psl.eu}}

\date{}

\begin{document}

\maketitle

% The biggest barriers to understanding how societies engaged with legends and histories in the Middle Ages are the unexplained gaps in the evidence. Through the creation and exchange of manuscripts, people developed and shared knights' tales across languages and generations. Manuscripts provide the primary evidence for this cultural transmission. However, looking at only extant manuscripts cannot reveal how many lost manuscripts are missing from the archive. This epistemic blindspot impedes scholars from fully visualizing how stories and textual traditions evolved through the centuries.

% The LostMa project seeks to leverage the gaps in the evidence in order to overcome some of them. Our hypothesis is that we can emulate the process of textual transmission through stochastic birth-and-death processes with multi-agent simulations. Such an artificial intelligence model requires as much accurate information as possible in order to optimize its predictions. Therefore, one of the project's secondary objectives, which will be of use to many other projects with their own primary objectives, is to assemble a massive corpus of chivalric literature. In focusing on textual traditions, rather than documents, our corpus includes extant manuscripts and fragments, whose text we can analyze, as well as missing documents, which we know or presume existed based on stemmata and other scholarly work.

This document explains and justifies the architecture of a data model whose objective is to help build knowledge about traditions of chivalric literature in the Middle Ages. The building blocks of this knowledge are Medieval works and their expressions in various text versions, as well as both extant and missing inscriptions of those versions on physical objects, including manuscripts and fragments. By structuring this information in a relational framework, linking abstract textual traditions and concrete archival evidence, the data model aims to let users curate large-scale collections of information from a diverse corpus, including Medieval French, Middle Dutch, and Old Norse.

The model needs to be able to execute the following user tasks:

\begin{itemize}
    \item Return a collection of resources (manuscripts, works, etc.) according to search criteria.
    \begin{enumerate}
        \item Ex. All the manuscripts that contain at least one witness of David Aubert's 1462 prosified version of \textit{Renaut de Montauban}. \textrightarrow [\textbf{Doc}\textsubscript{1}, \textbf{Doc}\textsubscript{2}, ... \textbf{Doc}\textsubscript{i}]
        \item Ex. All works that are grouped within the cycle about the knight Renaut de Montauban. \textrightarrow [\textbf{Work}\textsubscript{1}, \textbf{Work}\textsubscript{2}, ... \textbf{Work}\textsubscript{i}]
    \end{enumerate}
    \item Return a single resource according to search criteria.
    \begin{enumerate}
        \item Ex. The five-volume witness of David Aubert's version of \textit{Renaut de Montauban} that is partly conserved in Paris and partly in Munich. \textrightarrow [\textbf{Work}\textsubscript{x}]
        \item Ex. The earliest known version of the work \textit{Renaut de Montauban}. \textrightarrow [\textbf{Text}\textsubscript{y}]
    \end{enumerate}
    \item Return a tree of all the ancestors, descendents, and/or neighbors linked to a single resource or a collection of resources based on search criteria.
    \begin{enumerate}
        \item Ex. All the witnesses of David Aubert's version of \textit{Renaut de Montauban}, along with all the witnesses' page ranges as well as the bibliographic references to the physical documents that contain those pages. \textrightarrow \brackettext{
            \brackettext{\textbf{Wit}\textsubscript{1}, \brackettext{
                \textbf{Pag}\textsubscript{a}, \textbf{Doc}\textsubscript{a}
            }},
            \brackettext{\textbf{Wit}\textsubscript{2}, 
                \brackettext{
                    \textbf{Pag}\textsubscript{b}, \textbf{Doc}\textsubscript{b}
                },
                \brackettext{
                    \textbf{Pag}\textsubscript{c}, \textbf{Doc}\textsubscript{c}
                }
            },
            \ldots
        }
    \end{enumerate}
\end{itemize}

\section{Precedent Models}
The data model must be adaptable yet consistent. Users should be able to request collections of linked resources regardless the genre or language of the works involved. Furthermore, they should be able to combine repertories through shared vocabularies and metadata fields. As such, we face an age-old dilemna. We are pulled between being specific and idiosyncratic, on the one hand, and being simple and generalizable on the other. To confront this issue, we first turn to other successful data models. From those precedents, we adopt a hierarchical approach and certain ontological distinctions, notably distinctions between a \textit{Work} (the story), a \textit{Text} (a telling of the story), and a \textit{Witness} (a written version of that telling).

\subsection{Functional Requirements for Bibliographic Records}
One of the most robust bibliographic data models is the Functional Requirements for Bibliographic Records (FRBR), which slowly solidified during the last quarter of the twentieth century and is still today used to structure bibliographic databases.\footcite[][]{FRBR_Worldcat} In 1961, information scientists from around the world gathered in Paris to discuss best practices for cataloguing. Following their Paris meeting, the International Federation of Library Associations (IFLA) supported further discussions and members developed theoretical frameworks for organizing bibliographic data. Finally, in 1997, the IFLA approved the FRBR.\footcite{FRBR_1998} The FRBR articulates a generalizable hierarchy of four related entities: Works (\textbf{W}), Expressions (\textbf{E}), Manifestations (\textbf{M}), and Items (\textbf{I}). One of the examples in the 2009 corrected edition demonstrates how these entities relate to one another through the case of Johann Sebastian Bach's Goldberg variations.\footcite[We have augmented the example by adding an Item.][58]{FRBR_2009}

\begin{itemize}
    \item \textbf{W\textsubscript{1}} J. S. Bach's Goldberg variations
    \begin{itemize}
        \item \textbf{E\textsubscript{1}} performances by Glenn Gould recorded in 1981
        \begin{itemize}
            \item \textbf{M\textsubscript{1}} recording released on 33 1/3 rpm sound disc in 1982 by CBS Records
            \item \textbf{M\textsubscript{2}} recording re-released on compact disc in 1993 by Sony
            \begin{itemize}
                \item \textbf{I\textsubscript{1}} copy held at Cook Music Library, Bloomington, Indiana, USA (WOODWARD CD .B118 K1.988-35)
            \end{itemize}
        \end{itemize}
    \end{itemize}
\end{itemize}

In the FRBR model, a \textit{Work} is ``a distinct intellectual artistic creation'' and can have a title, a date, and a form or genre.\footcite[][17]{FRBR_2009} An \textit{Expression} is an ``intellectual or artistic realization'' of the \textit{Work} in some specific form, such as alpha-numeric or musical notation, sound, image, or movement.\footcite[][19]{FRBR_2009} Manifestations are the concrete, ``physical embodiment'' of the \textit{Expression} of a \textit{Work}.\footcite[][21]{FRBR_2009} For instance, if an \textit{Expression} of a \textit{Work} is notated in Vedic Sanskrit around 900 BCE, a \textit{Manifestation} of that literary \textit{Expression} would be a physical manuscript. Finally, an Item is a ``single exemplar of a manifestation,'' which is a concept best suited for mass-produced resources, such as copies of an edited book or, as in the example above, copies of a CD-ROM.\footcite[][24]{FRBR_2009}

% \begin{table}[ht]
% \begin{center}
    \begin{tabular}
        {|p{0.06\textwidth}|p{0.15\textwidth}|p{0.7\textwidth}|}
        \hline
        \textbf{Depth} & \textbf{FRBR Entity} & \textbf{Test Case} \\
        \hline
        1 & Work & One story of Renaut de Montauban and his brothers (\textit{The four sons of Aymon}).\\
        \hline
        2 & Expression & David Aubert’s prosified version of the story, known as \textit{Renaut de Montauban}.\\
        \hline
        3 & Manifestation & A copy of Aubert’s version, which was composed in five volumes in the 1460s.\\
        \hline
        4 & Document & One of the copy's volumes, i.e. Bibliothèque nationale de France, Arsenal, français, MS 5072.\\
        \hline
    \end{tabular}
\end{center}
% \caption{FRBR with test case \textit{Renaut de Montauban}.}
% \end{table}

% \subsection{\textit{Encoded Archival Standards}}
% The Encoded Archival Standards (EAD) is an international set of bibliographic protocols whose first definitive version, 1.0, was recently released in 2024. When encoded in EAD, the item's electronic record has two main components: information about the record itself (\texttt{<control>}), and information about the archival item (\texttt{<archdesc>}). The latter is relevant to the resources in LostMa's data model and includes the following sections:

\begin{itemize}
    \item Descriptive Identifier (\texttt{<did>}) : identifies the constitution of the archival item and other identifying information
    \begin{itemize}
        \item Title
        \item Author
        \item Date
        \item Repository
        \item Language
    \end{itemize}
    \item Scope Content (\texttt{<scopecontent>}) : meaningfully describes the content in the archival item
\end{itemize}

\subsection{\textit{Textrad}}
More recently this century, Patrick Sahle has developed a set of entity relationships that have significantly impacted the fields of philology and textual criticism.\footcite{Sahle2013} At the core of his wheel model (\textit{Textrad}), which we illustrate as a triangle in Figure \ref{fig:Textrad}, Sahle deconstructs the idea of a text into three primary dimensions: text as a story's ideal content \brackettext{\textit{Inhalt}}, not mediated by anything (\textbf{I}); text as a story's linguistic content \brackettext{\textit{sprachlichem Gehalt}}, mediated through human language (\textbf{S}); and text as a physical document \brackettext{\textit{Dokument}} (\textbf{D}) on which a story's content is materially represented.

\begin{figure}[ht]
\begin{center}
    \tikzstyle{s} = [rectangle, rounded corners, minimum width=3cm, minimum height=1cm,text centered, draw=black]
    \tikzstyle{arrow} = [thick,->,>=stealth]
    \begin{tikzpicture}[-,shorten >=1pt,auto,node distance=2.8cm,semithick]
    \tikzstyle{every state}=[fill=red,draw=none,text=white]
    \node[s] (D) {Document (\textbf{D})};
    \node[s] (I) [above right of=D, xshift=5em, yshift=5em] {Ideal Content (\textbf{I})};
    \node[s] (S) [below right of=I, xshift=5em, yshift=-5em] {Linguistic Content (\textbf{S})};
    \node [below=1cm,text width=8cm, xshift=1cm] at (S)
    {
    Text-as-linguistic-content (\textit{Sprache}): The expression of the text's content (\textit{Inhalt}, \textbf{I}) through human language.
    };
    \node [below=1cm,text width=8cm, xshift=-1cm] at (D)
    {
    Text-as-document (\textit{Dokument}): The physical object onto which the text's content (\textit{Inhalt}, \textbf{I}) is inscribed, through the medium of language.
    };
    \node [right=2cm,text width=7cm] at (I)
    {
    Text-as-content (\textit{Inhalt}): The text's ideal, unstructured content, which is subsequently articulated through language (\textit{Sprache}, \textbf{S}) and recorded in a material document (\textit{Dokument}, \textbf{D}).
    };
    \draw [arrow] (S) -> (D);
    \draw [arrow] (D) -> (I);
    \draw [arrow] (I) -> (S);
    \end{tikzpicture}
\end{center}
\caption{Main spokes of Sahle's \textit{Textrad}}
\label{fig:Textrad}
\end{figure}

Between the main spokes of the text-wheel (\textit{Textrad}), as seen in Figure \ref{fig:TextradAll}, Sahle includes three other dimensions: text as a set of signs (\textit{Zeichen}), text as a structured sequence of content (\textit{Werk}), and text as a version (\textit{Fassung}). Sahle's embedded dimensions make certain assertions about texts. For example, changes in a text's linguistic content (\textbf{S}), such as translating \textit{Beowulf} from Old English to contemporary Italian, will produce a new version (\textbf{F}) of the text, which will in turn produce a different physical document (\textbf{D}). However, changing \textit{Beowulf} from Old English to contemporary Italian does not necessarily change the work (\textbf{W}) \textit{Beowulf} itself, which still features the same organizing structure of the content of the \textit{Beowulf} story (\textbf{I}).

\begin{figure}[ht]
\tikzstyle{s} = [rectangle, rounded corners, minimum width=3cm, minimum height=1cm,text centered, draw=black, fill=white]
\tikzstyle{arrow} = [thick,->,>=stealth]
\begin{center}
\begin{tikzpicture}[-,shorten >=1pt,auto,node distance=2.8cm,semithick]
\tikzstyle{every state}=[fill=red,draw=none,text=white]
\node[s] (D) {Document (\textbf{D})};
\node[s] (I) [above right of=D, xshift=5em, yshift=5em] {Ideal Content (\textbf{I})};
\node[s] (S) [below right of=I, xshift=5em, yshift=-5em] {Linguistic Content (\textbf{S})};

\begin{pgfonlayer}{bg}    % select the background layer
    \draw [arrow] (S) -> (D);
    \draw [arrow] (D) -> (I);
    \draw [arrow] (I) -> (S);
\end{pgfonlayer}

\node[s] (W) [above left of=S] {Work (\textbf{W})};
\node[s] (Z) [left of=W, xshift=-2.5em] {Signs (\textbf{Z})};
\node[s] (F) [below right of=Z] {Version (\textbf{F})};

\node [below=1cm,xshift=2cm,text width=6cm] at (S)
{
Ex. The structured content of \textit{Beowulf} as mediated through the human language Old English, and to which an author or authors can be attributed.
};
\node [below=1cm,text width=4.5cm] at (F)
{
Ex. The sequence of letters that represents the linguistic content of \textit{Beowulf} in Old English.
};
\node [below=1cm,xshift=-1cm,text width=5cm] at (D)
{
Ex. A manuscript containing the Old-English version of \textit{Beowulf} (Nowell Codex).
};structued
\node [right=1cm,xshift=1cm,text width=5cm] at (W)
{
Ex. An organized structuring of the content in \textit{Beowulf}.
};
\node [right=1cm,xshift=1cm,text width=5cm] at (I)
{
Ex. Abstractly, the content of the story \textit{Beowulf}.
};

\end{tikzpicture}
\end{center}
\caption{All of Sahle's \textit{Textrad}}
\label{fig:TextradAll}
\end{figure}

Sahle avoids ascribing the term ``text'' to any one entity within the text-wheel. However, as Frédéric Duval notes, many scholars in the fields of philology, textual criticism, and scholarly editions habitually rely on the term ``text'' as well as ``work'' and ``document.'' Attempting to make explicit many scholars' and editors' implied typologies, Duval summarizes the state of the field as such:
\begin{quote}
    ``\brackettext{\textit{W}}\textit{ork} designates the author's text, eventually the text corresponding to the author's intention, and implies authenticity; \textit{text} denotes the linguistic sequence, which is attested in the document that is transmitting the work; finally \textit{document} is a physical manifestation of a text.''
    \footcite[``\textit{work} désigne le texte de l'auteur, éventuellement le texte correspondant à la volonté de l'auteur, et implique la notion d'authenticité ; \textit{text} dénomme la séquence linguistique attestée dans un document transmettant l'œuvre ; enfin \textit{document} est une manifestation physique d'un text''][16]{Duval2017}
\end{quote}

\noindent In Figure \ref{fig:DuvalTypes}, we overlay the typologies of Sahle's text-wheel with Duval's summary of the mainstream typology used in textual criticism and scholarly editing.

\begin{figure}[ht]
    
\tikzstyle{s} = [rectangle, rounded corners, minimum width=3cm, minimum height=1cm,text centered, draw=black, fill=white]
\tikzstyle{arrow} = [thick,->,>=stealth]
\begin{center}
\begin{tikzpicture}[-,shorten >=1pt,auto,node distance=2.8cm,semithick]
\tikzstyle{every state}=[fill=red,draw=none,text=white]
    \node[s] (D) {Document (\textbf{D})};
    \node[s] (I) [above right of=D, xshift=5em, yshift=5em] {Ideal Content (\textbf{I})};
    \node[s] (S) [below right of=I, xshift=5em, yshift=-5em] {Linguistic Content (\textbf{S})};

    \begin{pgfonlayer}{bg}    % select the background layer
        \draw [arrow] (S) -> (D);
        \draw [arrow] (D) -> (I);
        \draw [arrow] (I) -> (S);
    \end{pgfonlayer}

    \node[s] (W) [above left of=S] {Work (\textbf{W})};
    \node[s] (Z) [left of=W, xshift=-2.5em] {Signs (\textbf{Z})};
    \node[s] (F) [below right of=Z] {Version (\textbf{F})};

    % Duval notes
    \node (Sn) [below=1cm,xshift=3cm,text width=6cm] at (S)
    {
    \textbf{Work}: ``The author's text, eventually the text corresponding to the author's intention, and implies authenticity.''
    };
    \node (Fn) [below=1cm,text width=5cm] at (F)
    {
    \textbf{Text}: ``The linguistic sequence, which is attested in the document that is transmitting the work.''
    };
    \node (Dn) [below=1cm,xshift=-2cm,text width=4cm] at (D)
    {
    \textbf{Document}: ``A physical manifestation of a text.''
    };

    \draw [arrow] (Sn) -> (Fn);
    \draw [dashed, -] (Sn) -- (S);
    \draw [arrow] (Fn) -> (Dn);
    \draw [dashed, -] (Fn) -- (F);
    \draw [dashed, -] (Dn) -- (D);

\end{tikzpicture}
\end{center}
\caption{Overlap between Duval's summary and Sahle's \textit{Textrad}.}
\label{fig:DuvalTypes}
\end{figure}

In seeking to resolve terminological discrepancies between the fields of textual criticism and scholarly editing, Duval introduces a new discrepancy between his concept of \textit{work} and Sahle's \textit{Werke} concept. Given his focus on articulating the tripartite \textit{work}-\textit{text}-\textit{document} concerns of textual criticism and scholarly editions, this is not a problem for Duval. Sahle's concept of a work (\textit{Werke}), which is not yet mediated through any human language or literary style, is not typically the subject of scholarly editions or close textual readings. Such endeavors focus on what Sahle would call linguistic content (\textbf{S}).



\section{Precedent Models' Limitations}
To stress test the FRBR and \textit{Textrad}, neither of which were specially designed for Medieval literature, we explore two cases. The first is a work about the knight Renaut de Montauban. In the 1460s, the work was recomposed in a multi-volume manuscript, which introduces some complexity in relating a text version (\textit{Fassung}) to multiple documents. Second, we underscore the challenge of relating text versions to physical documents through the case of a lost manuscript, which, before being dismembered, allegedly transmitted a version of the \textit{Chanson d'Otinel} and a version of the \textit{Chanson d'Aspremont}. Parts of both are conserved today in two different nineteenth-century collections of Medieval fragments. Through the data model, we want to be able to recognize what scholars argue, which is that the fragments were once transmitted together in a now lost manuscript.

\subsection{\textit{Renaut de Montauban} and multi-volume witnesses}

Let us start with the easier case, concerning a work about the legendary knight Renaut de Montauban. Sahle's \textit{Werke} concept is helpful here in that it defines a work as the ordering of a story's abstract content (\textit{Inhalt}) into a narrative structure. The sequence of events whose order defines the \textit{Work} \textit{Renaut de Montauban} begins with a backstory that contextualizes the main conflict.\footcite[Whether this opening section constitutes a prologue, in alignment with the generic expectations of a prologue for \textit{chansons de geste}, is the subject of scholarly debate.][]{Leverage2000} The narrator explains that four brothers, Aymon, Beuves, Girart, and Doon, once fought together against the emperor Charlemagne. Beuves, who is duke of Aigremont and vassal of Charlemagne, refused some of the duties the emperor had imposed, which provokes the latter's fury. The brothers ccame to Beuves's aid and joined his conflict with Charlemagne. \textit{Renaut de Montauban} begins with this history because the work's main storyline focuses on how a new generation of brothers, Aymon's sons Renaut, Alard, Guichard, and Richard, again push back against Charlemagne and how the powerful emperor pursues revenge.\footnotemark\footnotetext{Having made peace, duke Aymon brings his four sons, Renaut, Alard, Guichard, et Richard, with him to meet emperor Charlemagne in Paris. After the meeting, Renaut plays chess with Charlemagne's nephew, but the game devolves into a dispute. The knight ultimately slays the nephew. Fearing the emperor's vengeful wrath, Renaut and his brothers flee Paris on the back of a magical horse, Bayard. Adventures ensue.}

In various languages and forms, many people have composed and recomposed the \textit{Work} known by its French title as \textit{Renaut de Montauban}. The earliest instance of such a composition, what Sahle calls the text-as-linguistic-content \brackettext{\textit{sprachlichen Gehalt}} and the FRBR call an \textit{Expression} of the \textit{Work}, dates from the end of the twelfth century and was expressed in French alexandrine verse. Following the argument Duval makes to rely on famililar terminology, but with greater attention to precise definitions, we use the term \textit{Text} to refer to the idea of the articulation of a \textit{Work} in some language and form, Sahle's text-as-linguistic-content.

Our test case concerns a \textit{Text} of \textit{Renaut de Montauban} that was written down about three centuries after the first known \textit{Text} was inscribed. While working for the Burgundian duke Philippe le Bon between 1459 and 1465, David Aubert adapted \textit{Renaut de Montauban} into contemporary prose. He structured his text's linguistic content inside evenly distributed chapters, each about 8 to 12 leaves long, and in five manuscript volumes. Each volume was about the same length, between 350 to 399 leaves, and featured nearly the same number of illuminations, between 47 and 53.\footcite[][]{Querel2007} Copies of those volumes exist in Paris and Munich; the first four are in the Bibliothèque nationale de France and the fifth volume is in the Bayerische Staatsbibliothek. Crucially, in the terminology of the \textit{Textrad}, this particular \textit{Version} (\brackettext{Fassung}) of Aubert's \textit{Text} was intentionally produced in five physical documents; it does not exist today in five manuscripts because of some process of deconstruction after its production.

Both the FRBR and Sahle's \textit{Textrad} are capable of modeling some of what we have described thus far. As Table \ref{tab:Renaut} demonstrates, both the FRBR and Sahle find that the term \textit{Work} appropriately describes the ordered series of episodes that define Renaut's story or revolt. We also use the term \textit{Work}. Regarding instances of the \textit{Work}, the FRBR and Sahle each prefer terms that point to a comparable idea of human language, \textit{Expression} and \textit{Sprache}, respectively. In our test case, this concept maps to the French-language content that is Aubert's version of the \textit{Work}. However, agin in line with Duval, we hold that the term \textit{Text} is better suited to indicate Aubert's specific telling, articulated in a specific human language and literary style, of the \textit{Work}. Finally, the text version or \textit{Fassung}, as Sahle puts it, is what we and many scholars of Medieval literature call a \textit{Witness} to the \textit{Text} that someone composed. In addition to adopting the term \textit{Text}, as Duval encourages, we also borrow the word \textit{Witness}, which is common in codicology, philology, and Medieval studies.

\begin{table}[ht]
    \begin{center}
    \begin{tabular}{|p{0.44\textwidth}|p{0.12\textwidth}|p{0.21\textwidth}|p{0.1\textwidth}|}
        \hline
        \textbf{Aspect} & \textbf{FRBR} & \textbf{\textit{Textrad}} & \textbf{LostMa} \\ \hline
        The subject matter of \textit{Renaut de Montauban}. & NA & text-as-content \newline (\textit{Inhalt}, \textbf{I}) & NA \\ \hline
        The ordered series of episodes about Renaut's revolt, starting with the backstory of the older generation. & \textit{Work} & text-as-work \newline (\textit{Werke}, \textbf{W}) & \textit{Work} \\ \hline
        David Aubert's prosified French-language version. & \textit{Expression} & text-as-linguistic-content \newline (\textit{Sprache}, \textbf{S}) & \textit{Text} \\ \hline
        A five-volume copy of the prose version. & \textit{Manifestation} & text-as-version \newline (\textit{Fassung}, \textbf{F}) & \textit{Witness} \\ \hline
    \end{tabular}
    \end{center}
\caption{FRBR and \textit{Textrad} modeling \textit{Renaut de Montauban}.}
\label{tab:Renaut}
\end{table}

\subsubsection{\textit{Text} v. \textit{Witness}}

What is the difference between the \textit{Text} and the \textit{Witness}, or the text-as-linguistic-content and the text-as-version, as Sahle puts it? The former is indifferent to versions of spelling and formatting, much in the way the \textit{Work} is indifferent to language and form. By transcribing here the first line of Aubert's \textit{Text}, we would either be choosing which \textit{Witness} to copy, as is the case with critical editions, or creating our own text version. Variations at the minute level of linguistic expression and dialect distinguish one text-as-version (\textit{Witness}) from another. For example, the second volume of two different multi-volume \textit{Witnesses} of Aubert's \textit{Text} each present the same linguistic content, as seen in Tabel \ref{tab:TextVersions}, but each one has its own way of transforming linguistic content into written language.

\begin{table}[ht]
    \begin{center}
        \begin{tabular}{|p{0.2\textwidth}|p{0.7\textwidth}|}
            \hline
            Manuscript shelfmark & First line \\
            \hline \hline
            BNF Arsenal 5073 & Qui a veu l'istoire de Maugis d'Aigremont poeut avoir leu comment Vivien. \\
            \hline
            BNF français 19174 & Qui a veue l'istoire de Maugis d'Aigremont bien au long, peult avoir veu comment Vivien.\\
            \hline
        \end{tabular}
    \end{center}
\caption{First lines of the second volume of two \textit{Witnesses} to Aubert's \textit{Text}, \textit{Renaut de Montauban}.}
\label{tab:TextVersions}
\end{table}

\noindent Both transcribed sequences of characters in Table \ref{tab:TextVersions} are each part of a different text-as-version or \textit{Witness}, both of which are part of the same text-as-linguistic-content or \textit{Text}, in this case, Aubert's \textit{Renaut de Montauban}.

\subsubsection{\textit{Cycle}}

Neither the FRBR nor the \textit{Textrad} have a concept well suited for describing a group of related \textit{Works}. Both models' ontological categories reach only to the extent of a single \textit{Work}. Yet when treating similar content, such as the adventures of Renaut de Montauban and his family, sometimes \textit{Works} of chivalric literature cohere into a collection known as a \textit{Cycle}. While the earliest instance of \textit{Renaut de Montauban} appeared in the twelfth century, other \textit{Works} were created in the centuries that followed, which revisited characters central in \textit{Renaut de Montauban}. As Gaëtan Augustine notes in his dissertation, the later \textit{Works} do not all return to the core \textit{Work's} central themes, namely revolt and imperial tyranny, but they neverthless form a \textit{Cycle} by building a world around figures central to the \textit{Work} \textit{Renaut de Montauban}.\footcite[][]{Augustine2020} The precedents, FRBR and \textit{Textrad}, are not equipped to model such metadata.

\subsubsection{The \textit{Archival Item} and its \textit{Pages}}

A more significant incomptability between the \textit{Renaut de Montauban} test case and the existing data models arises when we introduce archival evidence. The five-volume \textit{Witness} to Aubert's \textit{Text} that is partly preserved in Paris and partly in Munich does not directly relate to what Sahle calls the text-as-document (\textit{Dokument}) and what the FRBR call the \textit{Item}. Both concepts, especially in the FRBR, are meant to fully describe a uniquely produced object. However, the written \textit{Witness} to an author's \textit{Text} does not always share the same boundaries as an \textit{Item} in the archive.

We need an intervening entity to connect the multi-volume \textit{Witness} to the five phyiscal documents that substantiate it. Moreover, that intervening entity needs to have a discrete beginning and an end, representing one continous set of leafs or pages in the \textit{Archival Item}. Let us call the intervening entity \textit{Pages}, by which we mean one continous set of pages or leafs in a physical document. Figure \ref{fig:RenautFinal} illustrates how we would model the multi-volume \textit{Witness} of Aubert's \textit{Text}, taking into account the fact that the beginning and end of an \textit{Item} in the archive is not always synchronous with the beginning and end of a \textit{Witness}.

\begin{figure}[ht]
    \begin{center}
    \tikzstyle{s} = [rectangle, rounded corners, minimum width=2cm, text width=2cm, minimum height=1cm, text centered, draw=black]
\tikzstyle{arrow} = [thick,->,>=stealth]
\begin{tikzpicture}[-,shorten >=1pt,auto,node distance=2cm,semithick]
\tikzstyle{every state}=[fill=red,draw=none,text=white]

\node[s] (Wk) {\textit{Renaut de Montauban}};
\node [left=6cm, text width=3cm] at (Wk) {Work};

\node[s] (T) [below of=Wk] {Aubert's \textit{Renaut de Montauban}};
\node [left=6cm, text width=3cm] at (T) {Text};

\node[s] (W) [below of=T] {five-volume copy, 1460s};
\node [left=6cm, text width=3cm] at (W) {Witness};

\node[s] (V3) [below of=W] {1-350v (vol.3)};
\node[s] (V2) [left of=V3, xshift=-0.5cm] {1-353v (vol.2)};
\node[s] (V1) [left of=V2, xshift=-0.5cm] {1-395r (vol.1)};
\node[s] (V4) [right of=V3, xshift=0.5cm] {1-337v (vol.4)};
\node[s] (V5) [right of=V4, xshift=0.5cm] {1-399 (vol.5)};
\node [left=1cm, text width=3cm] at (V1) {Pages};

\node[s] (D1) [below of=V1] {BnF, Arsenal, M.S. 5072};
\node[s] (D2) [below of=V2] {BnF, Arsenal, M.S. 5073};
\node[s] (D3) [below of=V3] {BnF, Arsenal, M.S. 5074};
\node[s] (D4) [below of=V4] {BnF, Arsenal, M.S. 5075};
\node[s] (D5) [below of=V5] {BSB, Cod.gall. 7};
\node [left=1cm, text width=3cm] at (D1) {Archival Item};

\path[every node/.style={font=\sffamily\small}]
    (Wk) edge node [right] {} (T)
    (T) edge node [right] {} (W)
    (W) edge node [right] {} (V1)
    (W) edge node [right] {} (V2)
    (W) edge node [right] {} (V2)
    (W) edge node [right] {} (V3)
    (W) edge node [right] {} (V4)
    (W) edge node [right] {} (V5)
    (V1) edge node [right] {} (D1)
    (V2) edge node [right] {} (D2)
    (V3) edge node [right] {} (D3)
    (V4) edge node [right] {} (D4)
    (V5) edge node [right] {} (D5)
    ;

\end{tikzpicture}
    \caption{Provisionary model of \textit{Renaut de Montauban} case.}
    \label{fig:RenautFinal}
    \end{center}
\end{figure}

It is necessary to introduce the intervening \textit{Pages} entity between the \textit{Witness} and the \textit{Item}. The reason is that an archival \textit{Item} may contain more than one \textit{Witness}. This happens to not be the case with \textit{Renaut de Montauban}. Each of the \textit{Items} contains nothing but its part of the \textit{Witness}. However, our second test case does involve manuscripts that transmit \textit{Witness} of more than one \textit{Work}, which will demonstrate the need for further development of the preexisting models.

\subsection{\textit{Chanson d'Aspremont} and the lost manuscript}

In 1586, a notary named François Daunys certified an agreement among a group of people that reasserted who owned which properties in the French villages of Fournels, Le Mazet, and La Vachelerie. In the years that followed, notaries and others in the region kept track of the legal agreement, which eventually made its way into the Archives départementales de la Lozère. Around 1883, archivist Ferdinand André noticed that two old pieces of parchment, which were being used to cover the 1586 contract, were in fact fragments of \textit{chansons de geste}, which he presumed dated from the thirteenth-century. André communicated his discovery to his colleagues and the fragments, being separated from the legal document, were sent to the Bibliothèque nationale de France (BNF), where they are conserved today in a nineteenth-century collection of Medieval fragments.

The two pieces of parchment that André discovered make up the seventh and eighth folios of BNF, nouvelles aquisitions français (NAF) 5094. They are not of the same \textit{chanson}, though they do come from the same original manuscript. Thus, in our adapted version of the FRBR data model, informed by Sahle's \textit{Textrad}, the composite BNF NAF 5094 manuscript would upwardly relate to two \textit{Pages} entities, one which begins on 7r and ends on 7v, and the other which begins on 8r and ends on 8v. Each \textit{Pages} entity would then relate to its own \textit{Witness}, because each folio is the fragment of a different \textit{Text} of a different \textit{Work}. Figure \ref{fig:BNFNAF5094} illustrates these relationships.

\begin{figure}[ht]
    \begin{center}
        \tikzstyle{s} = [rectangle, rounded corners, minimum width=2cm, text width=3cm, minimum height=1cm, text centered, draw=black]
\tikzstyle{arrow} = [thick,->,>=stealth]
\begin{tikzpicture}[-,shorten >=1pt,auto,node distance=1.5cm,semithick]
\tikzstyle{every state}=[fill=red,draw=none,text=white]


\node[s] (WorkOtinel) {\textit{Chanson d'Otinel}};
\node[s] (WorkAspremont) [right of=WorkOtinel, xshift=3cm] {\textit{Chanson d'Aspremont}};
\node [left=2cm, text width=3cm] at (WorkOtinel) {Work};

\node[s] (ExpressionOtinel) [below of=WorkOtinel] {Anglo-Norman \textit{Chanson d'Otinel}};
\node[s] (ExpressionAspremont) [below of=WorkAspremont] {Anglo-Norman \textit{Chanson d'Aspremont}};
\node [left=2cm, text width=3cm] at (ExpressionOtinel) {Text};

\node[s] (ManifestationOtinel) [below of=ExpressionOtinel] {M};
\node[s] (ManifestationAspremontC) [below of=ExpressionAspremont] {P\textsubscript{4}};
% \node[s] (ManifestationAspremontP) [below of=ExpressionAspremont, xshift=2cm] {P\textsubscript{4}};
\node [left=2cm, text width=3cm] at (ManifestationOtinel) {Witness};

\node[s] (PagesOtinel) [below of=ManifestationOtinel] {fol. 7r-7v, part II};
\node[s] (PagesAspremontC) [below of=ManifestationAspremontC] {fol. 8r-8v, part II};
% \node[s] (PagesAspremontP) [below of=ManifestationAspremontP] {fol. 1r-2a};
\node [left=2cm, text width=3cm] at (PagesOtinel) {Pages};

\node[s] (ItemBNF) [below of=PagesOtinel, xshift=2.5cm] {Paris, BNF, NAF 5094};
% \node[s] (ItemCF) [below of=PagesAspremontP, xshift=-2cm] {Clermont-Ferrand, Arch. Dép., I F2};
\node[left=4.5cm, text width=3cm] at (ItemBNF) {Archival Item};

\path[every node/.style={font=\sffamily\small}]
    (WorkOtinel) edge node [right] {} (ExpressionOtinel)
    (ExpressionOtinel) edge node [right] {} (ManifestationOtinel)
    (ManifestationOtinel) edge node [right] {} (PagesOtinel)
    (PagesOtinel) edge node [right] {} (ItemBNF)
    (ItemBNF) edge node [right] {} (PagesAspremontC)
    (PagesAspremontC) edge node [right] {} (ManifestationAspremontC)
    (ManifestationAspremontC) edge node [right] {} (ExpressionAspremont)
    (ExpressionAspremont) edge node [right] {} (WorkAspremont)
    ;

% \path[every node/.style={font=\sffamily\small}]
%     (ExpressionAspremont) edge node [right] {} (ManifestationAspremontP)
%     (ManifestationAspremontP) edge node [right] {} (PagesAspremontP)
%     (PagesAspremontP) edge node [right] {} (ItemCF)
% ;

\end{tikzpicture}
    \end{center}
    \caption{Provisionary model of \textit{chansons de geste} in BNF NAF 5094.}
    \label{fig:BNFNAF5094}
\end{figure}

\subsubsection{The \textit{Attested Document}}

There is a glaring problem with the modeling in Figure \ref{fig:BNFNAF5094}. On the one hand, we know the Anglo-Norman \textit{Witnesses} of the two \textit{chansons de geste} are contained today in the same nineteenth-century collection of fragments, BNF NAF 5094, part II. This information is needed in order to locate digitisations of the \textit{Pages}' text, so it is good that the model includes it. On the other hand, however, we miss what scholars have compellingly argued, which is that these two \textit{chansons} were originally part of one composite manuscript. It is merely a coincidence that the two fragments (\textit{Witnesses} M and P\textsubscript{4}) are today in the same \textit{Archival Item}, BNF NAF 5094.

We need another entity, which we call \textit{Attested Document} and through which the two fragments will be reunited by virtue of scholarly argumentation and research that claim they were initially trasmitted together. More detail about our test case of the \textit{Chanson d'Aspremont} reveals why an \textit{Attested Document} entity is so crucial. In the 1880s, while François André was sharing his discovery about the \textit{Chanson d'Otinel} and \textit{Chanson d'Aspremont} fragments in the Archives départementales de la Lozère, archivist Paul Meyer noted that the archives of a nearby French \textit{département}, the Puy-de-Dôme, also had a fragment of the \textit{Chanson d'Aspremont} that was being used to cover some records in their collection. Unlike the Lozère archives, the Puy-de-Dôme archives did not give the Bibliothèque nationale de France their fragment of the \textit{Chanson d'Aspremont}. Today, one can consult it at the Archives départementales de Clermont-Ferrand under the shelfmark I F2.

The provisional data model illustrated Figure \ref{fig:AspremontCFBNF} allows us to recognize the attested coexistence of the three fragments in a now-lost manuscript. Because the missing document is not conserved as such anywhere, it lacks a clear identifier like the \textit{Archival Items} that have shelfmarks. To compensate, we generate a name that concatenates the \textit{Attested Document's} alleged parts. The known copies of the two \textit{chansons de geste}, \textit{Witnesses} M and P\textsubscript{4}, bypass their \textit{Pages} entities and meet one another directly in the \textit{Attested Document}.

\begin{figure}[ht]
    \begin{center}
        \tikzstyle{s} = [rectangle, rounded corners, minimum width=2cm, text width=3cm, minimum height=1cm, text centered, draw=black]
\tikzstyle{arrow} = [thick,->,>=stealth]
\begin{tikzpicture}[-,shorten >=1pt,auto,node distance=1.5cm,semithick]
\tikzstyle{every state}=[fill=red,draw=none,text=white]

\node[s] (Text) [] {Anglo-Norman \textit{Chanson d'Aspremont}};

\node[s] (P) [below of=Text, xshift=-3cm] {P\textsubscript{4}};
\node[s] (C) [below of=Text, xshift=3cm] {C};
\node[s] (M) [left of=P, xshift=-3cm] {M};
\node[s] (TextOtinel) [above of=M] {Anglo-Norman \textit{Chanson d'Otinel}};
\node [left=2cm, text width=3cm] at (TextOtinel) {Text};
\node [left=2cm, text width=3cm] at (M) {Witness};

\node[s] (WorkOtinel) [above of=TextOtinel] {\textit{Chanson d'Otinel}};
\node[s] (Work) [above of=Text] {\textit{Chanson d'Aspremont}};
\node [left=2cm, text width=3cm] at (WorkOtinel) {Work};

\node[s] (PagesP) [below of=P] {fol. 8r-8v};
\node[s] (PagesC) [below of=C] {fol. 1r-2a};
\node[s] (PagesM) [below of=M] {fol. 7r-7v};
\node [left=2cm, text width=3cm] at (PagesM) {Pages};

\node[s] (BNF) [below of=PagesP] {Paris, BNF NAF 5094};
\node[s] (CF) [below of=PagesC] {Clermont-Ferrand, Arch. Dép., I F2};
\node [left=6.5cm, text width=3cm] at (BNF) {Archival Item};

\node[s] (AttestedDoc) [below of=BNF, text width=5cm] 
{Paris, BNF NAF 5094, part II + (?) Clermont-Ferrand, I F2};
\node [left=5.5cm, text width=4cm] at (AttestedDoc) {Attested Document};

\path[every node/.style={font=\sffamily\small}]
  (Text) edge node [right] {} (P)
  (Work) edge node [right] {} (Text)
  (WorkOtinel) edge node [right] {} (TextOtinel)
  (Text) edge node [right] {} (C)
  (P) edge node [right] {} (PagesP)
  (C) edge node [right] {} (PagesC)
  (PagesP) edge node [right] {} (BNF)
  (PagesC) edge node [right] {} (CF)
  (P) edge[dashed, ->, bend left=90] node [left, yshift=-0.25cm] {was contained in} (AttestedDoc)
  (C) edge[dashed, ->, bend left=60] node [right, xshift=-0.5cm, yshift=-0.5cm] {was contained in} (AttestedDoc)
  (TextOtinel) edge node [right] {} (M)
  (M) edge node [right] {} (PagesM)
  (PagesM) edge node [right] {} (BNF)
  (M) edge[dashed, ->, bend right=45] node [right] {was contained in} (AttestedDoc)
  ;

\end{tikzpicture}
    \end{center}
    \caption{Provisionary model of \textit{chansons de geste} in BNF, NAF 5094 and Clermont-Ferrand I F2.}
    \label{fig:AspremontCFBNF}
\end{figure}

The model shown in Figure \ref{fig:AspremontCFBNF} perpetuates sigla (P\textsubscript{4}, C) that philologists have historically given to each fragment and communicates that both \textit{Witnesses} were likely transmitted together in the same manuscript (the \textit{Attested Document}). It does not, however, document the hypothesis that P\textsubscript{4} and C are in fact two parts of the same \textit{Witness}. While it is good to demonstrate that a manuscript contained copies of both the \textit{Chanson d'Otinel} and the \textit{Chanson d'Aspremont}, this is not sufficient. We do not want to lose the related theory that the verses of the Anglo-Norman \textit{Chanson d'Aspremont}, which are preserved today in two different fragments, are in fact different parts of one copy. Put another way, the model as is risks suggesting that the \textit{Attested Document} transmitted one copy of the Anglo-Norman \textit{Chanson d'Otinel} and two copies of the Anglo-Norman \textit{Chanson d'Aspremont}. The evidence does not suggest this latter claim.

\subsubsection{Inter-\textit{Witness} relationships}

It is not enough to fuse the \textit{Witnesses} P\textsubscript{4} and C into one. On the one hand, we want the data model to preserve the status of existing \textit{Witnesses} as well as operate with the sigla philologists have historically given to archival fragments. In other words, we want to keep the \textit{Witness} entities seen in Figure \ref{fig:AspremontCFBNF}. On the other hand, the data model needs to be able to reveal that the two fragmentary \textit{Witnesses}, P\textsubscript{4} and C, are presumably parts of one fragmented \textit{Witness}.

The P\textsubscript{4} \textit{Witness} in BNF NAF 5094 does not start at the beginning of the \textit{Text}, meaning the Anglo-Norman version of the \textit{Chanson d'Asprmont}. As Jean-Baptiste Camps writes in his dissertation, the Paris fragment is likely missing about 85 verses of the beginning, after which it presents 395 verses of the \textit{Text}.\footcite[][xcvi]{camps2016} The Clermont-Ferrand (C) fragment presents the end of the \textit{Text}; it would total 384 verses if not for edges of its pages being cut. In putting the two fragments together, they miss about 85 verses at the beginning and 9606 in the middle. Their content does not overlap and, based on handwriting and language, there is reason to believe they were produced as one copy (\textit{Witness}) of the Anglo-Norman \textit{Chanson d'Aspremont}.

\begin{table}[ht]
    \begin{center}
    \begin{tabular}{c||cccc}
        \textit{Witness} & initial lacuna & P\textsubscript{4} text & middle lacuna & C text \\
        \hline
        \hline
        P\textsubscript{4} + C & & 395 & & 377 [384] \\
        Ch & 84 & 415 & 9606 & 388
    \end{tabular}
    \end{center}
\caption{Reproduction of the comparison between the number of verses in the complete \textit{Witness} Ch and the fragments P\textsubscript{4} and C, originally published in the dissertation of Jean-Baptiste Camps\footcite[Table 1.3.][xcvii]{camps2016}}
\label{tab:CampsAspremont}
\end{table}

Why not create another entity, the \textit{Attested Witness}? An example of such an entity is suggested in the first row of Table \ref{tab:CampsAspremont}, bearing the name ``P\textsubscript{4} + C.'' There is an ontological difference between the \textit{Archival Item} and the \textit{Attested Document} that justifies the latter's creation. The \textit{Archival Item} is a physical object in the world, which can be photographed and damaged. The \textit{Attested Document} is an historical claim asserting that a text object was once produced and existed in the world, but has not been conserved as such. An \textit{Attested Witness} and a regular \textit{Witness} are ontologically similar in that they are both philological claims that the sequence of characters inscribed on the pages of some document intentionally represents the linguistic content of a \textit{Text}, or of a text-as-linguistic-content as Sahle puts it. The difference between the alleged \textit{Witness} ``P\textsubscript{4} + C'' and the \textit{Witnesses} P\textsubscript{4} and C is not categorical; it does not justify the creation of a new entity.

The more efficient way to make legible in the data model the theoretical \textit{Witness} ``P\textsubscript{4} + C'' is to build that information into attributes of the \textit{Witnesses} that constitute it. A \textit{Witness} relates to one \textit{Text} and either no \textit{Pages}, as in the case of hypothetical nodes in a stemma, or one or more \textit{Pages} entities. In the case of the \textit{Chanson d'Aspremont}, each \textit{Witness} entity relates to one \textit{Pages} entity. This is in contrast to the \textit{Witness} in the case of \textit{Renaut de Montauban}, which relates to more than one \textit{Pages} entities becaue it is a multi-volume \textit{Witness}.

The provisional model in Figure \ref{fig:WitnessRelations} shows how a \textit{Witness} can relate to another \textit{Witness} via the attribute ``is preceded by.'' In the \textit{Chanson d'Aspremont} case, the \textit{Witness} P\textsubscript{4} would not have any value assigned to the attribute ``is preceded by'' because it is the root of a sequence of \textit{Witness} fragments; it is not preceded by anything. The \textit{Witness} C, on the other hand, would refer to P\textsubscript{4} in its field ``is preceded by.'' The fact that the P\textsubscript{4} \textit{Witness} would not have any value in its ``is preceded by'' attribute, yet it has the status of ``fragment,'' could mean one of two things: (a) it is the only known fragment of that \textit{Witness}, or (b) that it is the root in a sequence of fragments that constitute one attested \textit{Witness}. To determine which is the case, one would need to identify roots of sequences by grouping all the \textit{Witnesses} based on other fragment \textit{Witnesses} referenced in the attribute ``is preceded by.''

\begin{figure}[ht]
    \begin{center}
        \tikzstyle{s} = [rectangle, rounded corners, minimum width=2cm, text width=3cm, minimum height=1cm, text centered, draw=black]
\tikzstyle{arrow} = [thick,->,>=stealth]
\begin{tikzpicture}[-,shorten >=1pt,auto,node distance=1.5cm,semithick]
\tikzstyle{every state}=[fill=red,draw=none,text=white]

\node[s] (Text) [] {Anglo-Norman \textit{Chanson d'Aspremont}};

\node[s] (P) [below of=Text, xshift=-3cm] {P\textsubscript{4}};
\node[s] (C) [below of=Text, xshift=3cm] {C};
\node[s] (M) [left of=P, xshift=-3cm] {M};
\node[s] (TextOtinel) [above of=M] {Anglo-Norman \textit{Chanson d'Otinel}};
\node [left=1cm, text width=3cm] at (TextOtinel) {Text};
\node [left=1cm, text width=3cm] at (M) {Witness};

\node[s] (PagesP) [below of=P] {fol. 8r-8v};
\node[s] (PagesC) [below of=C] {fol. 1r-2a};
\node[s] (PagesM) [below of=M] {fol. 7r-7v};
\node [left=1cm, text width=3cm] at (PagesM) {Pages};

\node[s] (BNF) [below of=PagesP] {Paris, BNF NAF 5094};
\node[s] (CF) [below of=PagesC] {Clermont-Ferrand, Arch. Dép., I F2};
\node [left=5.5cm, text width=3cm] at (BNF) {Archival Item};

\node[s] (AttestedDoc) [below of=BNF, xshift=3cm, text width=5cm] 
{Paris, BNF NAF 5094, part II + (?) Clermont-Ferrand, I F2};
\node [left=7.5cm, text width=4cm] at (AttestedDoc) {Attested Document};

\path[every node/.style={font=\sffamily\small}]
  (Text) edge node [right] {} (P)
  (Text) edge node [right] {} (C)
  (P) edge node [right] {} (PagesP)
  (C) edge node [right] {} (PagesC)
  (PagesP) edge node [right] {} (BNF)
  (PagesC) edge node [right] {} (CF)
  (P) edge[dashed, ->, bend left=30] node [right, xshift=-0.75cm] {was contained in} (AttestedDoc)
  (C) edge[dashed, ->, bend right=30] node [right] {} (AttestedDoc)
  (C) edge[dashed, ->] node [right, xshift=-0.75cm, yshift=0.25cm] {is preceded by} (P)
  (TextOtinel) edge node [right] {} (M)
  (M) edge node [right] {} (PagesM)
  (PagesM) edge node [right] {} (BNF)
  (M) edge[dashed, ->, bend right=30] node [left] {was contained in} (AttestedDoc)
  ;

\end{tikzpicture}
    \end{center}
    \caption{Relationships between \textit{Witnesses} and neighboring entities in provisional model.}
    \label{fig:WitnessRelations}
\end{figure}



\section{Our Model}
Having explored existing data models and tested their limitations against several case studies from the Middle Ages, we propose a model with the entities and relationships represented in Figure \ref{fig:ProposedEntities}. To begin, we present all the entities we believe necessary for a data model tasked with organizing and delivering information about texts and traditions in Medieval literature, particularly texts about chivalric tales and legends. Then, we list all the attributes each entity features. Compared to the former, this latter section is more likely to need further modifications based on the idiosyncratic qualities of different literary corpora.

\subsection{Intertextuality}

On the top level of Figure \ref{fig:ProposedEntities}, running from left to right, are the three abstract entities, \textit{Cycle}, \textit{Work}, and \textit{Text}, that manage information about intertextuality in our corpus. \textit{Works} and \textit{Cycles} can be nested together within the scope of a \textit{Cycle}, based on the \textit{Works'} narrative content; both entities have the attribute ``is part of,'' which points to a \textit{Cycle} entity. \textit{Works} can also be modeled on other works, as in the case of a new \textit{Work} compiling and reworking the episodes and characters from two or more pre-existing \textit{Works}. The data model also registers intertextuality amongst \textit{Texts}, which can be modeled on one another as translations, prosifications, abbreviations, elaborations, versifications, or other forms of adapting the expression (\textit{Text}) of a common \textit{Work}. The nature of a \textit{Text's} relationship to a model \textit{Text} can be inferred through differences and similarities in the two entities' attributes, such as their language and literary form.

\begin{figure}[httb!]
    \begin{center}
        \tikzstyle{s} = [rectangle, rounded corners, minimum width=2cm, text width=2cm, minimum height=1cm, text centered, draw=black]
\tikzstyle{arrow} = [thick,->,>=stealth]
\begin{tikzpicture}[-,shorten >=1pt,auto,node distance=2.5cm,semithick]
\tikzstyle{every state}=[fill=red,draw=none,text=white]

\node[s] (cycle) [] {Cycle};
\node[s] (work) [right of=cycle, xshift=3cm] {Work};
\node[s] (text) [right of=work, xshift=3cm] {Text};

\node[s] (person) [below of=text] {Person};

\node[s] (attestedWitness) [below of=person, yshift=0.5cm] {Attested Witness};

\node[s] (witness) [below of=work, yshift=-1cm] {Witness};

\node[s] (pages) [left of=witness, xshift=-3cm] {Pages};

\node[s] (item) [below of=pages, yshift=-1cm] {Archival Item};

\node[s] (repo) [right of=item, xshift=3cm] {Repository};

\node[s] (document) [right of=repo, xshift=3cm] {Attested Document};

% \draw[dashed, ->] (witness.-45) arc (200:475:8mm) 
%   node[pos=0.5, below] () {is preceded by};
\draw[dashed, ->] (witness.45) arc (0:180:8mm)
  node[above, pos=0.5] () {is preceded by};
\draw[dashed, ->] (witness.45) arc (0:180:8mm)
  node[above, yshift=0.25cm] () {is modeled on};
\draw[->] (cycle.45) arc (0:180:8mm) 
  node[above, yshift=0.25cm] () {is part of};
\draw[->] (work.45) arc (0:180:8mm) 
  node[above, yshift=0.25cm] () {is modeled on};
\draw[->] (work) -- (cycle)
  node[pos=0.5, above] () {is part of};
\draw[->] (text) -- (work)
  node[pos=0.5, above] () {is expression of};
\draw[->] (witness) -- (text)
  node[pos=0.66, left] () {is manifestation of};
\draw[dashed,->] (witness) -- (document)
  node[pos=0.5, left] () {was contained in};
\draw[->] (pages) -- (item)
  node[pos=0.5, right] () {is contained in};
\draw[->] (witness) -- (pages)
  node[pos=0.5, above] () {is visible on};
\draw[->] (item) -- (repo)
  node[pos=0.5, above] () {is conserved in};
\draw[->] (text.45) arc (0:180:8mm) 
  node[yshift=0.25cm, above] () {is modeled on};
\draw[->] (text) -- (person)
  node[pos=0.66, right] {is translated by};
\draw[->] (text) -- (person)
  node[pos=0.33, right] {is written by};
\draw[dashed, ->] (witness) -- (attestedWitness)
  node[pos=0.5, above] {is modeled on};
\draw[dashed, ->] (attestedWitness) -- (document)
  node[pos=0.5, right] {was contained in};

\path[every node]
  (attestedWitness) edge[dashed, ->, bend right=100] node [pos=0.5, right] {is instance of} (text)
  ;

\end{tikzpicture}
    \end{center}
\caption{Proposed Entities.}
\label{fig:ProposedEntities}
\end{figure}

\subsection{Archival Evidence}

Stemming from an extant \textit{Witness} are potentially four types of relationships to other entities. First and foremost, a \textit{Witness} must relate to one and only one \textit{Text}. By definition, a \textit{Witness} is an extant version of a \textit{Text's} linguistic content, spelled out in a sequence of characters, as demonstrated in Table \ref{tab:TextVersions}. Therefore, the \textit{Witness} must be the manifestation of a \textit{Text}. Second, the \textit{Witness}, by virtue of being extant, must be visible on the \textit{Pages} of an \textit{Archival Item}. When the version of a \textit{Text} has survived through fragments, one of the fragmented version's \textit{Witnesses} can relate to another fragment through the attribute ``is preceded by fragment.'' An example of this inter-\textit{Witness} relationship is demonstrated in the case of the \textit{Chanson d'Aspremont} and the Table \ref{tab:CampsAspremont}. Finally, an extant \textit{Witness} can have formerly been contained in a document that has not survived but to whose historical existence scholars attest based on philological and codicological evidence.

\textit{Pages} represents an uninterrupted set of folios 
%JBC: Uninterrupted could be a problem. There are cases of very messy manuscript, where a witness will be found on pages, 1,42,5-10,120-134 of a physical document -- I have on example in a Welsh Otinel
%KC: This is an issue... We still need some kind of entity that gives us the first and last page of the section of a digitized document, which we can use to extract images.
and, when digitized, images of an \textit{Archival Item} on which the text content of a \textit{Witness} can be read. In our data model, a \textit{Pages} entity, which depends on an extant \textit{Witness}, must be contained in an \textit{Archival Item}. The \textit{Archival Item} can relate to multiple \textit{Pages} entities.
%JBC: there can be some overlap, normally just at the start/end pages, like this witness is p. (or fol.) 1-240, and the other one starts at 240 to 320
They represent a unique set of leafs or pages in an \textit{Archival Item}, and they present the content of only one \textit{Witness}. Lastly, the \textit{Archival Item}, being an object one can consult and which continues to persist, contrary to the \textit{Attested Document}, must be conserved in a \textit{Repository}.

\subsection{Genealogy of a \textit{Text}}
\label{sub:Graph}

While the proposed data model manages metadata about intertextuality within the corpus, its relational framework does not register how \textit{Witnesses} derive from one another. This choice reflects an assumption about intertextuality, meaning \textit{Works'} and \textit{Texts'} models, and stemma, meaning \textit{Witness'} models. On the one hand, we assume a relatively consistent consensus has been reached on the models of \textit{Works} and \textit{Texts}. We presume scholars and archivists have largely accepted as historical fact the assertion that one \textit{Text} is a translation of another \textit{Text}, that a \textit{Work} is a compilation of other \textit{Works}, that one \textit{Text} is the prose version of another \textit{Text}, and so on. Through analyses of the content and literary form of extant \textit{Witnesses} that are examples of \textit{Texts} and \textit{Works}, such claims of genealogy and intertextuality tend not to attract as much skepticism as claims scholars posit about which scribe derived their \textit{Witness} (Sahle's text-as-version) from the example of an older \textit{Witness}.

Such stemmatological claims are crucial, and we want to model them for the corpus. However, in addition to establishing a relationship between two \textit{Witness} entities, it is also critical to cite the source of that attested dependence. Furthermore, we want to permit conflicting stemma without over complicating the model's relationships. While it is possible to model such connections in a relational framework, exploiting those connections to respond to users' queries risks pushing the model to become over complex. The choice becomes one between exploting one over-complicated model or harmonizing and jointly exploiting two simpler models.

Rather than reinvent the wheel and model stemma in a complex relational framework, we propose pairing the former with a graph framework, which philologists have used for decades.\footcite[][]{Zundert2020} The relational framework's main tasks are, therefore, to generate entities relevant to our corpus, store metadata about them, and establish intertextual relations at the level of the abstract, narrative content, meaning the \textit{Work} and \textit{Text}. In conjunction, the graph database's \textit{Witness} nodes will bear the same identifier as their counterpart in the relational database and thus can be enriched with metadata.

The project \textit{OpenStemmata} has already designed a workflow through which contributors can encode stemma that have been published in a scholarly context or otherwise submitted by scholars. For example, Giovanni Palumbo and Paolo Rinoldi, who, while developing a critical edition of the French \textit{Chanson d'Aspremont}, produced a stemma of the first part of the \textit{Text}. Their stemma was encoded and uploaded to the \textit{OpenStemmata} database, and we have represented it here in Figure \ref{fig:GraphFramework}. The graph's gray node P\textsubscript{4} is equivalent to the \textit{Witness} P\textsubscript{4} in our earlier \textit{Chanson d'Aspremont} test case, as seen in Figures \ref{fig:BNFNAF5094}, \ref{fig:AspremontCFBNF}, and \ref{fig:WitnessRelations}.

\begin{figure}[htb!]
    \begin{center}
        \tikzstyle{s} = [circle, minimum width=1cm, text width=1cm, minimum height=1cm, text centered, draw=black]
\tikzstyle{n} = [circle, minimum width=1cm, text width=1cm, minimum height=1cm, text centered, draw=black, fill=gray!30]
\tikzstyle{p} = [circle, minimum width=1cm, text width=1cm, minimum height=1cm, text centered, draw=black, line width=0.75mm, fill=gray!30]
\tikzstyle{arrow} = [thick,->,>=stealth]
\begin{tikzpicture}[-,shorten >=1pt,auto,node distance=2cm,semithick]
\tikzstyle{every state}=[fill=red,draw=none,text=white]

\node[s] (w) {$\omega$};

\node[s] (a) [below of=w, xshift=-5cm] {$\alpha$};
\node[n] (p2) [below of=a, xshift=-1cm] {P\textsubscript{2}};
\node[n] (p5) [below of=a, xshift=1cm] {P\textsubscript{5}};

\path[every node/.style={font=\sffamily\small}]
    (w) edge[->] node[]{} (a)
    (a) edge[->] node[]{} (p2)
    (a) edge[->] node[]{} (p5)
;

\node[s] (b) [below of=w] {$\beta$};
\node[n] (W) [below of=b, xshift=-2cm] {W};
\node[n] (p1) [below of=b] {P\textsubscript{1}};
\node[s] (bb) [below of=b, yshift=-1cm, xshift=1cm] {};
\node[n] (B) [below of=bb, xshift=-1cm] {B};
\node[n] (R) [below of=bb, xshift=1cm] {R};

\path[every node/.style={font=\sffamily\small}]
    (w) edge[->] node[]{} (b)
    (b) edge[->] node[]{} (W)
    (b) edge[->] node[]{} (p1)
    (b) edge[->, bend left=15] node[]{} (bb)
    (bb) edge[->] node[]{} (B)
    (bb) edge[->] node[]{} (R)
;

\node[s] (d) [below of=w, xshift=5cm] {$\delta$};
\node[n] (l3) [below of=d, yshift=-1cm, xshift=-2cm] {L\textsubscript{3}};
\node[s] (dd1) [right of=l3] {};
\node[n] (ch) [below of=dd1, xshift=-1cm] {Ch};
\node[p] (p4) [below of=dd1, xshift=1cm] {P\textsubscript{4}};
\node[s] (dd2) [below of=d, xshift=2cm, yshift=-5cm] {};
\node[n] (p3) [below of=dd2, xshift=-1cm] {P\textsubscript{3}};
\node[s] (x) [below of=dd2, xshift=1cm] {\textit{x}};
\node[n] (v6) [below of=x, xshift=-1cm] {V\textsubscript{6}};
\node[s] (y) [below of=x, xshift=1cm] {\textit{y}};
\node[n] (cha) [below of=y, xshift=-1cm] {Cha};
\node[n] (v4) [below of=y, xshift=1cm] {V\textsubscript{4}};

\path[every node/.style={font=\sffamily\small}]
    (w) edge[->] node[]{} (d)
    (d) edge[->] node[]{} (l3)
    (d) edge[->] node[]{} (dd1)
    (dd1) edge[->] node[]{} (ch)
    (dd1) edge[->] node[]{} (p4)
    (d) edge[->, bend left=15] node[]{} (dd2)
    (dd2) edge[->] node[]{} (p3)
    (dd2) edge[->] node[]{} (x)
    (x) edge[->] node[]{} (v6)
    (x) edge[->] node[]{} (y)
    (y) edge[->] node[]{} (cha)
    (y) edge[->] node[]{} (v4)
;

\end{tikzpicture}

% \node[s] (w) {$\omega$};
% \node[s] (wl) [below of=w, xshift=-2cm] {};
% \node[s] (a) [below of=w, xshift=2cm] {$\alpha$};
% \node[s] (b) [below of=wl, xshift=-4cm] {$\beta$};
% \node[s] (gamma) [below of=wl] {$\gamma$};
% \node[n] (W) [below of=b, xshift=-2cm] {W};
% \node[n] (p5) [below of=b] {P\textsubscript{5}};
% \node[s] (gammal) [below of=gamma] {};
% \node[n] (l3) [below of=gammal, xshift=-2cm] {L\textsubscript{3}};
% \node[n] (ch) [below of=gammal] {Ch};
% \node[s] (x) [below of=gamma, xshift=2cm] {\textit{x}};
% \node[s] (y) [below of=x] {\textit{x}};
% \node[n] (cha) [below of=y, xshift=-1cm] {Cha};
% \node[n] (v4) [below of=y, xshift=1cm] {V\textsubscript{4}};
% \node[n] (v6) [below of=x, xshift=2cm] {V\textsubscript{6}};
% \node[n] (p2) [below of=a] {P\textsubscript{2}};
% \node[n] (p3) [below of=a, xshift=2cm] {P\textsubscript{3}};
% \node[n] (l2) [below of=a, xshift=4cm] {L\textsubscript{2}};

% \path[every node/.style={font=\sffamily\small}]
%     (w) edge[->] node[]{} (a)
%     (w) edge[->] node[]{} (wl)
%     (wl) edge[->] node[]{} (b)
%     (wl) edge[->] node[]{} (gamma)
%     (b) edge[->] node[]{} (W)
%     (b) edge[->] node[]{} (p5)
%     (gamma) edge[->] node[]{} (gammal)
%     (gamma) edge[->] node[]{} (x)
%     (gammal) edge[->] node[]{} (l3)
%     (gammal) edge[->] node[]{} (ch)
%     (x) edge[->] node[]{} (y)
%     (x) edge[->] node[]{} (v6)
%     (y) edge[->] node[]{} (cha)
%     (y) edge[->] node[]{} (v4)
%     (a) edge[->] node[]{} (p2)
%     (a) edge[->] node[]{} (p3)
%     (a) edge[->] node[]{} (l2)
% ;

% \end{tikzpicture}
    \end{center}
\caption{Stemma of extant \textit{Witnesses} (gray) and hypothetical intermediaries (white)
%JBC: in strict terminological claims, since they are hypothetic, they are not witnesses: a witness must bear witness, so it must exist and be observable. Those hypothetic nodes are usually called the hypothetical intermediaries [between witnesses and the original], sub-archetype and archetype (for branches head, and tradition head, so, in this case, $\alpha$, $\beta$, $\delta$; and $\omega$ respectively) and original. This would deserve at least a note.
of the first part of the French \textit{Chanson d'Aspremont} by G. Palumbo and P. Rinoldi.}
\label{fig:GraphFramework}
\end{figure}

We propose outsourcing the data model's \textit{Text} genealogies to the \textit{OpenStemmata} project's established workflow, encoding standards, and open-source repository. Subsequently, we reconcile the \textit{Witness} nodes in the \textit{OpenStemmata} database, further valorizing and enriching those publications, with records registered in our proposed relational data model. Finally, having linked the two record types, such as the \textit{Witness} P\textsubscript{4} of the Anglo-Norman \textit{Chanson d'Aspremont}, the data model can deliver thoroughly enriched and linked records based on users' requests.

\subsection{Reconciling and citing records}

Finally, we propose a \textit{Reference} table (see Figure \ref{fig:ProposedEntitiesReference}), which associates certain entities with bibliographic resources. This last addition serves two purposes. First, it associates an entity with a unique identifier in an authoritative aggregator, such as WikiData, Biblissima, and VIAF (Virtual International Authority File). This association helps reconcile potentially duplicate records in the data model and helps reconcile records between the relational data model and an \textit{OpenStemmata} graph. Second, \textit{Reference} associates entities with scholarly citations. In addition to making the data model interoperable with other databases, including VIAF, the \textit{Reference} entity also allows us to enrich records with a scholarly bibliography.

\begin{figure}[htb!]
    \begin{center}
        \tikzstyle{s} = [rectangle, rounded corners, minimum width=2cm, text width=2cm, minimum height=1cm, text centered, draw=black]
\tikzstyle{arrow} = [thick,->,>=stealth]
\begin{tikzpicture}[-,shorten >=1pt,auto,node distance=2.5cm,semithick]
\tikzstyle{every state}=[fill=red,draw=none,text=white]

\node[s] (cycle) [] {Cycle};
\node[s] (work) [right of=cycle, xshift=3cm] {Work};
\node[s] (text) [right of=work, xshift=3cm] {Text};

\node[s] (witness) [below of=work, yshift=-1cm] {Witness};

\node[s] (person) [right of=witness, xshift=3cm] {Person};

% \node[s] (attestedWitness) [below of=person, yshift=0.5cm] {Attested Witness};

\node[s] (pages) [left of=witness, xshift=-3cm] {Pages};

\node[s] (item) [below of=pages, yshift=-1cm, xshift=2cm] {Archival Item};

\node[s] (repo) [right of=item, xshift=3cm] {Repository};

\node[s] (document) [right of=repo, xshift=1cm] {Attested Document};

\node[s] (reference) [above of=pages, xshift=-2cm, yshift=-0.75cm] {Reference};

% \draw[dashed, ->] (witness.-45) arc (200:475:8mm) 
%   node[pos=0.5, below] () {is preceded by};
\draw[->] (witness.45) arc (0:180:8mm)
  node[above, yshift=0.25cm] () {is preceded by};
% \draw[dashed, ->] (witness.45) arc (0:180:8mm)
%   node[above, yshift=0.25cm] () {is modeled on};
\draw[->] (cycle.45) arc (0:180:8mm) 
  node[above, yshift=0.25cm] () {is part of};
\draw[->] (work.45) arc (0:180:8mm) 
  node[above, yshift=0.25cm] () {is modeled on};
\draw[->] (work) -- (cycle)
  node[pos=0.5, above] () {is part of};
\draw[->] (text) -- (work)
  node[pos=0.5, above] () {is expression of};
\draw[->] (witness) -- (text)
  node[pos=0.66, left] () {is manifestation of};
\draw[->] (witness) -- (document)
  node[pos=0.5, left] () {was contained in};
\draw[->] (pages) -- (item)
  node[pos=0.5, right] () {is contained in};
\draw[->] (witness) -- (pages)
  node[pos=0.5, above] () {is visible on};
\draw[->] (item) -- (repo)
  node[pos=0.5, above] () {is conserved in};
\draw[->] (text.45) arc (0:180:8mm) 
  node[yshift=0.25cm, above] () {is modeled on};
\draw[->] (text) -- (person)
  node[pos=0.75, above, xshift=-0.75cm] {is translated by};
\draw[->] (text) -- (person)
  node[pos=0.33, below, xshift=-0.75cm] {is written by};
% \draw[dashed, ->] (witness) -- (attestedWitness)
%   node[pos=0.5, above] {is modeled on};
% \draw[dashed, ->] (attestedWitness) -- (document)
%   node[pos=0.5, right] {was contained in};

\path[every node]
  (reference) edge[dashed, ->] node[]{} (cycle)
  (reference) edge[dashed, ->] node[]{} (work)
  (reference) edge[dashed, ->] node[]{} (text)
  (reference) edge[dashed, ->, bend left=10] node[]{} (person)
  (reference) edge[dashed, ->, bend right=30] node[]{} (item)
  (reference) edge[dashed, ->] node[]{} (witness)
  (reference) edge[dashed, ->, bend right=76] node[]{} (document)
  ;

% \path[every node]
%   (attestedWitness) edge[dashed, ->, bend right=100] node [pos=0.5, right] {is instance of} (text)
%   ;

\end{tikzpicture}
    \end{center}
\caption{Proposed Entities with \textit{Reference} table.}
\label{fig:ProposedEntitiesReference}
\end{figure}

\textit{Reference} can reconcile conflicting \textit{Cycle}, \textit{Work}, \textit{Text}, \textit{Person}, and \textit{Archival Item} entities and link them to universal unique identifiers. The \textit{Archival Item} entity should ideally have an Archival Resource Key provided by its \textit{Repository}, which should help avoid duplicate records of the same manuscript. The entities \textit{Cycle}, \textit{Work}, \textit{Person}, and \textit{Text} overlap with entities in the VIAF database; our model's \textit{Person} is equivalent to VIAF's \textit{Personal Names} and our model's \textit{Cycle}, \textit{Work}, \textit{Text} records can find an equivalent under the VIAF's \textit{Work} and \textit{Expression} records. The latter two entities in the VIAF model are borrowed from the FRBR.

For example, the \textit{Work} \textit{Renaut de Montauban}, seen in Table \ref{sub:Work}, has the VIAF identifier {\texttt{174185484}}, as seen in the last row of Table \ref{sub:Reference}. By linking a \textit{Work} record to a \textit{Reference} record, which in this case associates \textit{Renaut de Montauban} with an identifier in the VIAF database, we can enrich the \textit{Work} with all the linked data available in VIAF. Furthermore, while our record for \textit{Renaut de Montauban} has one title, its link to the VIAF database via the \textit{Reference} entity associates the \textit{Work} with other names by which people might identify it, including \textit{Renaud de Montauban}, \textit{Reinolt von Montelban}, and \textit{Quatre fils Aymon}, further improving the data's interoperability and resilience against duplication.

\begin{figure}[htb!]

    \begin{subfigure}{\textwidth}
        \begin{center}
        \begin{tabular}{|p{0.05\textwidth}|p{0.2\textwidth}|p{0.1\textwidth}|}
            \hline
            \textbf{ID} & \textbf{title} & \textbf{is part of} \\ \hline
            1 & Renaut de Montauban & \\ \hline
        \end{tabular}
        \end{center}
    \subcaption{\textit{Cycle} record for \textit{Renaut de Montauban}.}
    \label{sub:Cycle}
    \vspace*{1em}
    \end{subfigure}

    \begin{subfigure}{\textwidth}
        \begin{center}
        \begin{tabular}{|p{0.05\textwidth}|p{0.2\textwidth}|p{0.1\textwidth}|p{0.15\textwidth}|}
            \hline
            \textbf{ID} & \textbf{title} & \textbf{is part of} & \textbf{is modeled on} \\ \hline
            2 & Renaut de Montauban & 1 & \\ \hline
        \end{tabular}
        \end{center}
    \subcaption{\textit{Work} record for \textit{Renaut de Montauban}.}
    \label{sub:Work}
    \vspace*{1em}
    \end{subfigure}

    \begin{subfigure}{\textwidth}
        \begin{center}
        \begin{tabular}{|p{0.05\textwidth}|p{0.05\textwidth}|p{0.1\textwidth}|p{0.1\textwidth}|p{0.2\textwidth}|p{0.3\textwidth}|}
            \hline
            \textbf{entity type} & \textbf{entity ID} & \textbf{unique identifier} & \textbf{identifier source} & \textbf{permalink} & \textbf{citation} \\ \hline
            Cycle & 1 & & & & \cite{Augustine2020} \\ \hline
            Cycle & 1 & 5045 & Arlima & \url{https://arlima.net/no/5045} & \\ \hline
            Cycle & 1 & Q59212800 & WikiData & \url{https://www.wikidata.org/wiki/Q59212800} & \\ \hline
            Work & 2 & 318 & Arlima & \url{https://arlima.net/no/318} & \\ \hline
            Work & 2 & Q115962675 & WikiData & \url{https://www.wikidata.org/wiki/Q115962675} & \\ \hline
            Work & 2 & 174185484 & VIAF & \url{http://viaf.org/viaf/174185484} & \\ \hline
        \end{tabular}
        \end{center}
    \subcaption{\textit{Reference} records for \textit{Renaut de Montauban}.}
    \vspace*{1em}
    \label{sub:Reference}
    \end{subfigure}

\label{tab:ReferenceRenaut}
\end{figure}


% \subsubsection{Relational Database Framework}
% To realize this conceptual hierarchy in a relational database framework, we created an intermediary entity between the \textit{Witness} and the \textit{Document}. A relational entity (table) between the two core concepts is necessary to accommodate such cases as the Renaut de Montauban witness. Because the \textit{Document} entity is indifferent to the content it contains, the entity does not bare any information about on which of its pages the text is inscribed. However, that information cannot consistently be attributed to the \textit{Witness} entity either because, as in the case of Renaut de Montauban, sometimes a single witness is manifest across multiple passages in multiple documents. Therefore, to maintain the relationship between a potentially multi-volume or fragmented witness, like ``Am'' of \textit{Renaut de Montauban}, we need an intermediary entity, which we call \textit{Pages}.

The entity-relationship diagram in Figure \ref{fig:ERDiagram} illustrates the five core entities already discussed, the intermediary \textit{Pages}, and two others: \textit{Repository} and \textit{Stemma}. The former is effectively a controlled vocabulary for the different archival institutions in which manuscripts can be stored, but realized as an entity in order to feature attributes that help contributors better identify the institution and avoid adding duplicates. As a base for the \textit{Repository} table, we have scraped, cleaned, and enriched all the institutions listed on the Archives Portal Europe website.\footcite[][]{ArchivesPortal}

The \textit{Stemma} entity also functions as a kind of relational table between \textit{Text} and \textit{Witness}, linking both resources to a stemma publication in the \textit{OpenStemmata Project} database. The actual, hereditary relationships between \textit{Witnesses} of a \textit{Text} are encoded in a graph database. The relational database framework, shown in Figure \ref{fig:ERDiagram}, primarily documents the existence of \textit{Witneses} and \textit{Texts} in one of the encoded stemmata.

\begin{figure}[ht]
\begin{center}
\begin{tikzpicture}
    \pic [] { entity = {{cycle}
        {\textbf{Cycle}}
        {
          \textbf{id} \\
          \hline
          \textbf{parent\_cycle} \\
          \hline
          title \\
          \hline
          arlima\_id \\
          jonas\_id \\
          deaf\_ref \\
          moisan\_ref
        }
    }};
  
    \pic [right = 4em of cycle] { entity = {{work}
        {\textbf{Work}}
        {
          \textbf{id} \\
          \hline
          \textbf{cycle\_id} \\
          \hline
          title \\
          \hline
          arlima\_id \\
          jonas\_id \\
          deaf\_ref \\
          moisan\_ref
        }
    }};
  
    \draw[oone-many] (cycle) -- node[label,above]{} (work);
    
    \pic [right = 4em of work] { entityassociative = {{text}
        {\textbf{Text}}
        {
          \textbf{id} \\
          \hline
          \textbf{work\_id} \\
          \hline
          title \\
          responsibility \\
          earliest\_witness \\
          \hline
          genre \\
          subgenre \\
          language \\
          form \\
          verse\_type \\
          rhyme\_type \\
          length \\
          \hline
          arlima\_id \\
          bossuat\_ref \\
          deaf\_ref \\
          jonas\_id \\
          moisan\_ref \\
          philobiblon\_id \\
          suard\_ref \\
          note\_reference
        }
    }};
  
    \draw[one-many] (work) -- node[label,above]{} (text);

    \pic [right = 4em of text] { entityassociative = {{witness}
        {\textbf{Witness}}
        {
          \textbf{id} \\
          \hline
          \textbf{text\_id} \\
          \hline
          siglum \\
          status \\
          \hline
          tei\_document\_id
        }
    }};

    \draw[one-omany] (text) -- node[label,above]{} (witness);

    \pic [below = 8em of text] { entityassociative = {{stemma}
        {\textbf{Stemma}}
        {
          \textbf{id} \\
          \hline
          \textbf{witness\_id} \\
          \textbf{text\_id} \\
          \hline
          author \\
          citation
        }
    }};

    \draw[one-omany] (text) -- node[label,above]{} (stemma);
    \draw[many-omany] (witness) -- node[label,above]{} (stemma);

    \pic [below = 6em of witness] { entityassociative = {{pages}
        {\textbf{Pages}}
        {
          \textbf{id} \\
          \hline
          \textbf{witness\_id} \\
          \textbf{document\_id} \\
          \hline
          volume\_order \\
          page\_start \\
          page\_end \\
          view\_start \\
          view\_end \\
          iiif\_manifest \\
        }
    }};

    \draw[one-omany] (witness) -- node[label,above]{} (pages);

    \pic [below = 2em of work] { entityassociative = {{doc}
        {\textbf{Historical Document}}
        {
          \textbf{id} \\
          \hline
          \textbf{repository\_id} \\
          collection \\
          shelfmark \\
          note\_provenance \\
          origin\_location \\
          origin\_date \\
          date\_freetext \\
          date\_source \\
          \hline
          iiif\_manifest \\
          digitisation\_url \\
          \hline
          arlima\_id \\
          bnf\_id \\
          jonas\_id \\
          philobiblon\_id \\
          viaf\_id \\
          note\_reference
        }
    }};

    \draw[many-one] (pages) -- node[label,above]{} (doc);

    \pic [left = 4em of doc] { entityassociative = {{repo}
        {\textbf{Repository}}
        {
          \textbf{id} \\
          \hline
          Site Name \\
          Street Address \\
          Website \\
          \hline
          Archives Portal Europe Reference \\
          \hline
          Country \\
          WikiData ID\\
          GeoShape Map
        }
    }};

    \draw[many-one] (doc) -- node[label,above]{} (repo);

  \end{tikzpicture}
\caption{LostMa Entity-Relationship Diagram}
\label{fig:ERDiagram}
\end{center}
\end{figure}


% \section{Entities}

% \subsection{Repository}
% The \textit{Repository} is the archival institution in which a \textit{Historical Document} is conserved. It denotes the modern institution currently holding an extant document. Every \textit{Historical Document} must have a singular relationship to a \textit{Repository}. However, as is often the case, a single \textit{Repository} can be related to multiple \textit{Historical Documents}.

\subsubsection{Vocabularies}

The \textit{Repository} entity has only one controlled vocabulary, which is the modern-day name of the country in which the institution currently resides. This vocabulary is developed from the default list of countries in a Heurist database, augmented with the names of several missing countries, including England and Wales.

\subsubsection{Attributes}

\begin{longtable}{|
    |m{0.15\textwidth}
    |m{0.05\textwidth}
    |p{0.15\textwidth}
    |m{0.1\textwidth}
    |m{0.2\textwidth}
    |m{0.2\textwidth}
||}
    \hline
    Attribute Name & Count & Data Type & Vocab. & Description & Semantic Reference \\
    \hline

    \multicolumn{6}{|c|}{General Info}\\
    \hline
    \textbf{Site Name} %Field
        & 1 %N
        & single line (text)%Type
        & %Vocab
        & \textit{The Name of the Site.} %Description
        & %Semantic Reference
        \\
    \hline
    \textbf{Street Address} %Field
        & \[\leq 1\] %N
        & multi-line (text)%Type
        & %Vocab
        & \textit{Street address as recorded in Archives Portal Europe.} %Description
        & \texttt{street address}, \url{https://www.wikidata.org/wiki/Q24574749}%Semantic Reference
        \\
    \hline
    \textbf{Website} %Field
        & \[\leq 1\] %N
        & single line (text)%Type
        & %Vocab
        & \textit{Website of the archival institution according to Archives Portal Europe..} %Description
        & \texttt{official website}, \url{https://www.wikidata.org/wiki/Q22137024}%Semantic Reference
        \\
    \hline

    \multicolumn{6}{|c|}{Identification}\\
    \hline
    \textbf{Archives Portal Europe Reference} %Field
        & \[\leq 1\] %N
        & single line (text)%Type
        & %Vocab
        & \textit{URL to archival institution's information on Archives Portal Europe.} %Description
        & %Semantic Reference
        \\
    \hline

    \multicolumn{6}{|c|}{Country}\\
    \hline
    \textbf{Country} %Field
        & \[\leq 1\] %N
        & terms list (text)%Type
        & \textbf{Country}%Vocab
        & \textit{Modern-day country in which the repository resides.} %Description
        & \texttt{country}, \url{https://www.wikidata.org/wiki/Property:P17}%Semantic Reference
        \\
    \hline
    \textbf{WikiData ID} %Field
        & \[\leq 1\] %N
        & single line (text)%Type
        & %Vocab
        & \textit{WikiData unique identifier.} %Description
        & \texttt{Wikidata Q identifier}, \url{https://www.wikidata.org/wiki/Q43649390}%Semantic Reference
        \\
    \hline
    \textbf{GeoNames ID} %Field
        & \[\leq 1\] %N
        & numeric (integer)%Type
        & %Vocab
        & \textit{The ID of the place in GeoNames. Info is then available at \url{http://www.geonames.org/<geonameID>}} %Description
        & \texttt{GoeNames}, \url{https://www.wikidata.org/wiki/Q830106}%Semantic Reference
        \\
    \hline
    \textbf{GeoShape Map} %Field
        & \[\leq 1\] %N
        & single line (text)%Type
        & %Vocab
        & \textit{GeoShape map file from WikiData.} %Description
        & \texttt{geoshape}, \url{https://www.wikidata.org/wiki/Property:P3896}%Semantic Reference
        \\
    \hline
\caption{Proposed Repository Attributes} % needs to go inside longtable environment
\label{tab:proposedRepositoryAttributes}
\end{longtable}

% \subsection{Historical Document}
% The \textit{Historical Document} is the physical object that contains all or part of a \textit{Witness} through passages delimited in the related entity \textit{Pages}. It is imperative to avoid duplicated entries of the same \textit{Historical Document}, which is why we have included multiple identification fields linking the \textit{Historical Document} record to a unique record in another database, such as the Arlima database, the Bibliothèque nationale de France, the Jonas database, the Philobiblon database, and the VIAF (Virtual International Authority File) database. We are aware that by allowing contributors to identify a \textit{Historical Document} by any one of multiple references, we risk having the same manuscript entered in the database twice, once using one reference and again using another. Through data record linking techniques and querying the set of refefences, we have a methodology to remove duplicates. Finally, having assembled a unified set of \textit{Historical Documents}, we will eventually improve the Biblissima+ database and other projects' reciliation of historical records.

\subsubsection{Vocabularies}

The only controlled vocabulary for \textit{Historical Document} is the \textbf{Scripta}, which indicates the place where the document was transcribed. The \textbf{Scripta} vocabulary controls the value entered in the data field \textit{location\_of\_creation}.\footnotemark\footnotetext{The use of this vocabulary to control the location of the document's creation seems incorrect to me. Indeed, I think we should create a \textit{script} data field and use the \textbf{Scripta} vocabulary to control it. I also think we should change the \textit{location\_of\_creation} data type to a free text field. The reason to control the location field is to perform aggregates and analyses on it, but (a) many \textit{Historical Documents} will not have location information and (b) preemptively controlling the kind of possible location information seems a task more difficult than it is valuable at the data entry stage.}

    \begin{itemize}
        \item Picardy
        \item \brackettext{Needs development}
    \end{itemize}

\subsubsection{Attributes}

\begin{longtable}{|
    |m{0.15\textwidth}
    |m{0.05\textwidth}
    |p{0.15\textwidth}
    |m{0.1\textwidth}
    |m{0.2\textwidth}
    |m{0.2\textwidth}
||}
    \hline
    Attribute Name & Count & Data Type & Vocab. & Description & Semantic Reference \\
    \hline

    \multicolumn{6}{|c|}{Identification}\\
    \hline
    \textbf{repository\_id} %Field
        & 1 %N
        & \textbf{repository} (foreign key)%Type
        & %Vocab
        & \textit{The repository in which the historical document is currently or was last conserved.} %Description
        & \texttt{repository}, \url{https://www.wikidata.org/wiki/Q2145117}%Semantic Reference
        \\
    \hline
    \textbf{collection} %Field
        & \[\leq 1\] %N
        & single line (text)%Type
        & %Vocab
        & \textit{Name of the repository's collection which contains the historical document.} %Description
        & \texttt{archival collection}, \url{https://www.wikidata.org/wiki/Q9388534}%Semantic Reference
        \\
    \hline
    \textbf{shelfmark} %Field
        & 1 %N
        & single line (text)%Type
        & %Vocab
        & \textit{Shelfmark of the historical document in the repository.} %Description
        & \texttt{shelfmark}, \url{https://data.biblissima.fr/w/Property:P195} %Semantic Reference
        \\
    \hline
    \textbf{note \_provenance} %Field
        & \[\geq 0\] %N
        & multi-line (text) %Type
        & %Vocab
        & \textit{Note(s) concerning traditions and/or prior shelfmarks of historical resources.} %Description
        & %Semantic Reference
        \\
    \hline
    \multicolumn{6}{|c|}{Origin}\\
    \hline
    \textbf{location\_of \_creation} %Field
        & \[\leq 1\] %N
        & terms list (text)%Type
        & \textbf{Scripta}%Vocab
        & \textit{Place where the historical document was created.} %Description
        & \texttt{location of creation}, \url{https://www.wikidata.org/wiki/Property:P1071}%Semantic Reference
        \\
    \hline
    \textbf{creation\_date} %Field
        & \[\leq 1\] %N
        & date (list[date])%Type
        & %Vocab
        & \textit{The single or principal date of the resource's origin (may also include a range and/or fuzzy limits).} %Description
        & \texttt{inception}, \url{https://data.biblissima.fr/w/Property:P58}%Semantic Reference
        \\
    \hline
    \textbf{date\_freetext} %Field
        & \[\leq 1\] %N
        & single line (text)%Type
        & %Vocab
        & \textit{Date as it was stated in the reference.} %Description
        & \texttt{stated as}, \url{https://data.biblissima.fr/w/Property:P93}%Semantic Reference
        \\
    \hline
    \textbf{date\_source} %Field
        & \[\leq 1\] %N
        & single line (text, url)%Type
        & %Vocab
        & \textit{URL or citation of the reference from which the date was retrieved.} %Description
        & \texttt{retrieved from}, \url{https://data.biblissima.fr/w/Property:P168}%Semantic Reference
        \\
    \hline
    \multicolumn{6}{|c|}{Digitisation} \\
    \hline
    \textbf{iiif\_manifest} %Field
        & \[\leq 1\] %N
        & media file (link) %Type
        & %Vocab
        & \textit{IIIF manifest (JSON file) representing the digitisation of the resource.} %Description
        & \texttt{IIIF manifest}, \url{https://data.biblissima.fr/w/Property:P196} %Semantic Reference
        \\
    \hline
    \textbf{digitisation\_url} %Field
        & \[\leq 1\] %N
        & single line (text)%Type
        & %Vocab
        & \textit{Link to the resource's digitisation in a digital collection.} %Description
        & \texttt{digitzed at URL}, \url{https://data.biblissima.fr/w/Property:P197} %Semantic Reference
        \\
    \hline
    \multicolumn{6}{|c|}{References} \\
    \hline
    \textbf{arlima\_id} %Field
        & \[\leq 1\] %N
        & numeric (integer)%Type
        & %Vocab
        & \textit{Identifier in Arlima.} %Description
        & \texttt{ARLIMA ID}, \url{https://data.biblissima.fr/w/Property:P121} %Semantic Reference
        \\
    \hline
    \textbf{biblissima\_id} %Field
        & \[\leq 1\] %N
        & single line (text)%Type
        & %Vocab
        & \textit{Identifier in Biblissima+ database.} %Description
        & \texttt{Biblissima ID}, \url{https://data.biblissima.fr/w/Property:P129} %Semantic Reference
        \\
    \hline
    \textbf{bnf\_id} %Field
        & \[\leq 1\] %N
        & single line (text)%Type
        & %Vocab
        & \textit{Archival Resource Key from the Bibliothèque nationale de France.} %Description
        & \texttt{BnF ID}, \url{https://data.biblissima.fr/w/Property:P109}
        \\
    \hline
    \textbf{jonas\_id} %Field
        & \[\leq 1\] %N
        & numeric (integer)%Type
        & %Vocab
        & \textit{Identifier in Jonas.} %Description
        & \texttt{Jonas ID}, \url{https://data.biblissima.fr/w/Property:P140} %Semantic Reference
        \\
    \hline
    \textbf{philobiblon\_id} %Field
        & \[\leq 1\] %N
        & single line (text)%Type
        & %Vocab
        & \textit{Identifier in Philobiblon.} %Description
        & %Semantic Reference
        \\
    \hline
    \textbf{viaf\_id} %Field
        & \[\leq 1\] %N
        & single line (text)%Type
        & %Vocab
        & \textit{Identifier for the Virtual International Authority File database (VIAF).} %Description
        & \texttt{VIAF ID}, \url{https://data.biblissima.fr/w/Property:P113}%Semantic Reference
        \\
    \hline
    \textbf{note\_reference} %Field
        & \[\geq 0\] %N
        & multi-line (text)%Type
        & %Vocab
        & \textit{Note(s) concerning how to identify the entity in an external reference or database.} %Description
        & %Semantic Reference
        \\
    \hline

\caption{Proposed Historical Document Attributes} % needs to go inside longtable environment
\label{tab:proposedHistoricalDocumentAttributes}
\end{longtable}

% \subsection{Pages}
% The \textit{Pages} entity indicates one of the potentially multiple passages of consecutive text that compose the \textit{Witness}. For example, if a \textit{Witness} is fragmentary and parts are rebound in three disconnected passages of a manuscript, the \textit{Witness} would have a relationship to three \textit{Pages} entities, each of which would have a relationship to the same \textit{Historical Document} entity. On the other hand, if a \textit{Witness} is comprised of four manuscript volumes, the \textit{Witness} would be related to four \textit{Pages} entities, and each \textit{Pages} would be related to a different \textit{Historical Document}. Finally, as is most often the case, if a \textit{Witness} exists in one contiguous passage in one manuscript, it would have one relationship to a \textit{Pages} entity, which in turn would have a relationship to one \textit{Historical Document}. The intermediary \textit{Pages} entity allows for a diversity of \textit{Witnesses}, some of which might be composed across disconnected passages.

\subsubsection{Vocabularies}

The \textit{Pages} entity has no need for a controlled vocabulary because it primarily serves to store a single page range and, when a IIIF digitisation is available, metadata including the range of the page views and a link to a IIIF manifest file.

\subsubsection{Attributes}

\begin{longtable}{|
    |m{0.15\textwidth}
    |m{0.05\textwidth}
    |p{0.15\textwidth}
    |m{0.1\textwidth}
    |m{0.2\textwidth}
    |m{0.2\textwidth}
||}
    \hline
    Attribute Name & Count & Data Type & Vocab. & Description & Semantic Reference \\
    \hline

    \multicolumn{6}{|c|}{References}\\
    \hline
    \textbf{witness\_id} %Field
        & 1 %N
        & \textbf{witness} (foreign key)%Type
        & %Vocab
        & \textit{Reference to the witness to which the passage belongs.}%Description
        & %Semantic Reference
        \\
    \hline
    \textbf{document\_id} %Field
        & 1 %N
        & \textbf{document} (foreign key)%Type
        & %Vocab
        & \textit{Reference to the document in which the passage appears.}%Description
        & %Semantic Reference
        \\
    \hline

    \multicolumn{6}{|c|}{General Information}\\
    \hline
    \textbf{volume\_order} %Field
        & \[\geq 0\] %N
        & numeric (integer)%Type
        & %Vocab
        & \textit{If the witness is diffused across multiple volumes or fragments, the index of this volume or passage within the sequence.}%Description
        & %Semantic Reference
        \\
    \hline
    \textbf{page\_start} %Field
        & \[\geq 0\] %N
        & single line (text)%Type
        & %Vocab
        & \textit{Folio indication of the start of the textual content.}%Description
        & %Semantic Reference
        \\
    \hline
    \textbf{page\_end} %Field
        & \[\geq 0\] %N
        & single line (text)%Type
        & %Vocab
        & \textit{Folio indication of the end of the textual content.}%Description
        & %Semantic Reference
        \\
    \hline
    \textbf{view\_start} %Field
        & \[\geq 0\] %N
        & single line (text)%Type
        & %Vocab
        & \textit{If digitised, the first page view of the textual content.}%Description
        & %Semantic Reference
        \\
    \hline
    \textbf{view\_end} %Field
        & \[\geq 0\] %N
        & single line (text)%Type
        & %Vocab
        & \textit{If digitised, the last page view of the textual content.}%Description
        & %Semantic Reference
        \\
    \hline
    \textbf{iiif\_manifest} %Field
        & \[\geq 0\] %N
        & single line (text)%Type
        & %Vocab
        & \textit{URL to IIIF manifest of the images containing the textual content.}%Description
        & \texttt{IIIF manifest URL}, \url{https://www.wikidata.org/wiki/Property:P6108}%Semantic Reference
        \\
    \hline

\caption{Proposed Content Attributes} % needs to go inside longtable environment
\label{tab:proposedContentAttributes}
\end{longtable}

% \subsection{Witness}
% The \textit{Witness} is a realized instance of someone's text (\textit{Text}). Frédéric Duval, what we call the \textit{Witness} is a ``the linguistic sequence, which is attested in the document transmitting the work.''\footcite[``la séquence linguistique attestée dans un document transmettant l'oeuvre.''][16]{Duval2017} 


and what he calls a \textit{Fassung} (Version) is ``a specific sequence of characters''\footcite[``einen Bestand an Schriftzeichen.''][47]{Sahle2013} 

\subsubsection{Vocabularies}



\subsubsection{status\_wit}
\begin{itemize}
    \item citation
    \item complete
    \item fragment
    \item translation or rewriting
\end{itemize}

\subsubsection{Attributes}

\begin{longtable}{|
    |m{0.15\textwidth}
    |m{0.05\textwidth}
    |p{0.15\textwidth}
    |m{0.1\textwidth}
    |m{0.2\textwidth}
    |m{0.2\textwidth}
||}
    \hline
    Attribute Name & Count & Data Type & Vocab. & Description & Semantic Reference \\
    \hline

    \multicolumn{6}{|c|}{Relations}\\
    \hline
    \textbf{text\_id} %Field
        & 1 %N
        & \textbf{text} (foreign key)%Type
        & %Vocab
        & \textit{Reference to the text to which it is a witness.}%Description
        & %Semantic Reference
        \\
    \hline
    \multicolumn{6}{|c|}{General Information}\\
    \hline
    \textbf{siglum} %Field
        & \[\leq 1\] %N
        & single line (text)%Type
        & %Vocab
        & \textit{Philological identifier of the witness.}%Description
        & %Semantic Reference
        \\
    \hline
    \textbf{siglum\_ref} %Field
        & \[\leq 1\] %N
        & single line (text)%Type
        & %Vocab
        & \textit{Reference in which the siglum is used.}%Description
        & %Semantic Reference
        \\
    \hline
    \textbf{status} %Field
        & 1 %N
        & terms list (text)%Type
        & \textbf{status\_wit}%Vocab
        & \textit{Status of the witness.}%Description
        & \texttt{condition}, \url{https://www.wikidata.org/wiki/Q813912}%Semantic Reference
        \\
    \hline
    \textbf{tei\_document} %Field
        & \[\leq 1\] %N
        & single line (text)%Type
        & %Vocab
        & \textit{Reference pointing to the TEI document that depicts the witness's text.}%Description
        & \texttt{TEI/XML}, \url{https://www.wikidata.org/wiki/Q124622467}%Semantic Reference
        \\
    \hline

\caption{Proposed Witness Attributes} % needs to go inside longtable environment
\label{tab:proposedWitnessAttributes}
\end{longtable}

% \subsection{Text}
% 
\subsection{Vocabularies}

\subsubsection{genre}
\begin{itemize}
    \item epic
\end{itemize}

\subsubsection{subgenre}
\begin{multicols}{3}

\begin{itemize}
    \item Bestiaire
    \item biblique
    \item chanson de geste
    \item chanson de saint
    \item Chronicle
    \item Conte
    \item didactique
    \item drama
    \item Encyclopédie
    \item Fable
    \item hagiographie
    \item Lai
    \item lyrique
    \item Mircales+lyrique
    \item narratif
    \item roman
    \item roman antique
    \item roman antique en vers
    \item roman arthurien en prose
    \item roman arthurien en vers
    \item roman d'aventures
    \item roman en prose
    \item Vie
\end{itemize}

\end{multicols}

\subsubsection{Verse Type}
\begin{itemize}
    \item alexandrines
    \item decasyllables
    \item hexasyllables
    \item octosyllables
\end{itemize}

\subsubsection{Rhyme Type}
\begin{itemize}
    \item assonance
    \item rhyme
\end{itemize}

\subsubsection{Language}
\begin{itemize}
    \item frm
    \item fro
    \item fro\_ENG
    \item fro\_IT
    \item fro\_PRO
    \item pro
\end{itemize}

\subsection{Attributes}

\begin{longtable}{|
    |m{0.15\textwidth}
    |m{0.05\textwidth}
    |m{0.1\textwidth}
    |m{0.1\textwidth}
    |m{0.2\textwidth}
    |m{0.3\textwidth}
||}
    \hline
    Attribute Name & Count & Data Type & Vocab. & Description & Semantic Reference \\
    \hline

    \multicolumn{6}{|c|}{Relations}\\
    \hline
    \textbf{work\_id} %Field
        & 1 %N
        & \textbf{work} (foreign key)%Type
        & %Vocab
        & \textit{Reference to the work.}%Description
        & %Semantic Reference
        \\
    \hline
    
    \multicolumn{6}{|c|}{General Information}\\
    \hline
    \textbf{title} %Field
        & 1 %N
        & single line (text)%Type
        & %Vocab
        & \textit{Title of the text.}%Description
        & %Semantic Reference
        \\
    \hline
    \textbf{responsibility} %Field
        & \[\geq 0\] %N
        & \textbf{Person} (foreign key)%Type
        & %Vocab
        & \textit{Author or authors to whom the text is attributed.}%Description
        & %Semantic Reference
        \\
    \hline
    \textbf{earliest\_witness} %Field
        & \[\leq 1\] %N
        & date %Type
        & %Vocab
        & \textit{Date of the earliest identified or presumed witness, whether existant or not, which is effectively the origin date of the text.}%Description
        & %Semantic Reference
        \\
    \hline
    \textbf{genre} %Field
        & \[\leq 1\] %N
        & terms list (text)%Type
        & \textbf{genre}%Vocab
        & \textit{Genre of the text.}%Description
        & %Semantic Reference
        \\
    \hline
    \textbf{subgenre} %Field
        & \[\leq 1\] %N
        & terms list (text)%Type
        & \textbf{subgenre}%Vocab
        & \textit{Subgenre of the text.}%Description
        & %Semantic Reference
        \\
    \hline
    \textbf{language} %Field
        & \[\leq 1\] %N
        & terms list (text)%Type
        & \textbf{language}%Vocab
        & \textit{Primary language of the text.}%Description
        & %Semantic Reference
        \\
    \hline
    \textbf{form} %Field
        & \[\leq 1\] %N
        & single line (text)%Type
        & %Vocab
        & %Description
        & %Semantic Reference
        \\
    \hline
    \textbf{verse\_type} %Field
        & \[\leq 1\] %N
        & terms list (text)%Type
        & %Vocab
        & %Description
        & %Semantic Reference
        \\
    \hline
    \textbf{rhyme\_type} %Field
        & \[\leq 1\] %N
        & terms list (text)%Type
        & %Vocab
        & %Description
        & %Semantic Reference
        \\
    \hline
    \textbf{length} %Field
        & \[\leq 1\] %N
        & numeric (integer)%Type
        & %Vocab
        & %Description
        & %Semantic Reference
        \\
    \hline

    \multicolumn{6}{|c|}{Reference}\\
    \hline
    \textbf{arlima\_id} %Field
        & \[\leq 1\] %N
        & numeric (integer)%Type
        & %Vocab
        & \textit{Identifier in Arlima.} %Description
        & \texttt{ARLIMA ID}, \url{https://data.biblissima.fr/w/Property:P121} %Semantic Reference
        \\
    \hline
    \textbf{bossuat\_ref} %Field
        & \[\leq 1\] %N
        & single line (text)%Type
        & %Vocab
        & %Description
        & %Semantic Reference
        \\
    \hline
    \textbf{deaf\_ref} %Field
        & \[\leq 1\] %N
        & single line (text)%Type
        & %Vocab
        & %Description
        & %Semantic Reference
        \\
    \hline
    \textbf{jonas\_id} %Field
        & \[\leq 1\] %N
        & numeric (integer)%Type
        & %Vocab
        & \textit{Identifier in Jonas.} %Description
        & \texttt{Jonas ID}, \url{https://data.biblissima.fr/w/Property:P140} %Semantic Reference
        \\
    \hline
    \textbf{moisan\_id} %Field
        & \[\leq 1\] %N
        & single line (text)%Type
        & %Vocab
        & %Description
        & %Semantic Reference
        \\
    \hline
    \textbf{philobiblon\_id} %Field
        & \[\leq 1\] %N
        & single line (text)%Type
        & %Vocab
        & %Description
        & %Semantic Reference
        \\
    \hline
    \textbf{suard\_id} %Field
        & \[\leq 1\] %N
        & single line (text)%Type
        & %Vocab
        & %Description
        & %Semantic Reference
        \\
    \hline
    \textbf{note\_reference} %Field
        & \[\geq 0\] %N
        & multi-line (text)%Type
        & %Vocab
        & \textit{Note(s) concerning how to identify the entity in an external reference or database.} %Description
        & %Semantic Reference
        \\
    \hline

\caption{Proposed Text Attributes} % needs to go inside longtable environment
\label{tab:proposedTextAttributes}
\end{longtable}

% \section{Added}

% \begin{longtable}{|
%     |m{0.15\textwidth}
%     |m{0.05\textwidth}
%     |m{0.1\textwidth}
%     |m{0.1\textwidth}
%     |m{0.2\textwidth}
%     |m{0.3\textwidth}
% ||}
%     \hline
%     Attribute Name & Count & Data Type & Vocab. & Description & Semantic Reference \\
%     \hline

%     \multicolumn{6}{|c|}{Relations}\\
%     \hline
%     \textbf{note\_id} %Field
%         & \[\geq 0\] %N
%         & %Type
%         & %Vocab
%         & %Description
%         & %Semantic Reference
%         \\
%     \hline

% \caption{Proposed Attributes} % needs to go inside longtable environment
% \label{tab:proposedAttributes}
% \end{longtable}

\end{document}