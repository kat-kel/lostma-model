\tikzstyle{s} = [rectangle, rounded corners, minimum width=2cm, text width=3cm, minimum height=1cm, text centered, draw=black]
\tikzstyle{arrow} = [thick,->,>=stealth]
\begin{tikzpicture}[-,shorten >=1pt,auto,node distance=1.5cm,semithick]
\tikzstyle{every state}=[fill=red,draw=none,text=white]

\pic [] { entityassociative = {{pages}
{\textbf{Pages}}
{
    \textbf{ID} \\ \hline
    is contained in \\
    archival item part \\
    witness part \\
    folio start \\
    folio end \\
    image start \\
    image end \\
}
}};

\pic [right = 12em of pages] { entityassociative = {{item}
{\textbf{Archival Item}}
{
  \textbf{ID} \\
  \hline
  is conserved in\\
  collection\\
  shelfmark\\
  belatedly compiled\\
  provenance\\
  digitisation URL\\
  IIIF manifest\\
}
}};

\pic [right = 12em of item] { entityassociative = {{repository}
{\textbf{Repository}}
{
  \textbf{ID} \\
  \hline
  institutional name\\
  street address\\
  website\\
  AEP reference\\
  country\\
  WikiData ID\\
  GeoNames ID\\
  GeoShape Map\\
}
}};


\draw[omany-one] (pages) --node[label, above]{Archival Item contains} (item);
\draw[omany-one] (pages) --node[label, below]{0, 1, or many Pages} (item);

\draw[omany-one] (item) --node[label, above]{Archival item is conserved} (repository);
\draw[omany-one] (item) --node[label, below]{1 and only 1 Repository} (repository);

\end{tikzpicture}