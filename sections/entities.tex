\subsection{Cycle}

Definition: General theme that a group of \textit{Works} can share.  

\vspace{1em}
\noindent Attributes:
\begin{itemize}
    \item \texttt{title} (text, req., uniq.): Received name of the \textit{Cycle}, either in the language of the first known \textit{Text} to treat the matter or in the language most used in scholarship.
    \item \texttt{is part of} (foreign key, \textbf{Cycle}, opt., uniq.): The meta-\textit{Cycle}, of which the \textit{Cycle} is a part.
\end{itemize}

\begin{figure}[ht]
    \begin{center}
        \tikzstyle{s} = [rectangle, rounded corners, minimum width=2cm, text width=3cm, minimum height=1cm, text centered, draw=black]
\tikzstyle{arrow} = [thick,->,>=stealth]
\begin{tikzpicture}[-,shorten >=1pt,auto,node distance=1.5cm,semithick]
\tikzstyle{every state}=[fill=red,draw=none,text=white]

\pic [] { entityassociative = {{cycle}
{\textbf{Cycle}}
{
  \textbf{ID} \\
  \hline
  title \\
  is part of \\
}
}};

\pic [right = 10em of cycle] { entityassociative = {{work}
{\textbf{Work}}
{
  \textbf{ID} \\
  \hline
  title \\
  is part of \\
  is modeled on\\
}
}};

\pic [left = 10em of cycle] { entityassociative = {{reference}
{\textbf{Reference}}
{
  entity type\\
  entity ID \\
  unique identifier \\
  identifier source \\
  permalink\\
  citation \\
}
}};

\draw[omany-omany] (reference) -- node[label, above]{Cycle has 0, 1, or} (cycle);
\draw[omany-omany] (reference) -- node[label, below]{many References} (cycle);

\draw[one-many] (cycle.east) -- node[label, above]{Cycle has 1 or more} (work.west);
\draw[one-many] (cycle.east) -- node[label, below]{Works part of it}(work.west);

\end{tikzpicture}
    \end{center}
\label{fig:CycleER}
\caption{\textit{Cycle} entity relationships.}
\end{figure}

%%%%%%%%%%%%%%%%%%%%%%%%%

\subsection{Work}

Definition: Content of a story, which has a recognizable structure and can be recounted in different ways while remaining the same story.

\vspace{1em}
\noindent Attributes:
\begin{itemize}
    \item \texttt{title} (text, req., uniq.): Received name of the \textit{Work}, either in the language of the first known \textit{Text} to treat the matter or in the language most used in scholarship.
    \item \texttt{is part of} (foreign key, \textbf{Cycle}, opt., uniq.): The \textit{Cycle}, of which the \textit{Work} is a part.
    \item \texttt{is modeled on} (foreign key, \textbf{Work}, opt., repeat.): If a reworking of an anterior \textit{Work}, the \textit{Work} on which it is modeled.
\end{itemize}


\begin{figure}[ht]
    \begin{center}
        \tikzstyle{s} = [rectangle, rounded corners, minimum width=2cm, text width=3cm, minimum height=1cm, text centered, draw=black]
\tikzstyle{arrow} = [thick,->,>=stealth]
\begin{tikzpicture}[-,shorten >=1pt,auto,node distance=1.5cm,semithick]
\tikzstyle{every state}=[fill=red,draw=none,text=white]

\pic [] { entityassociative = {{cycle}
{\textbf{Cycle}}
{
  \textbf{ID} \\
  \hline
  title \\
  is part of \\
}
}};

\pic [right = 12em of cycle] { entityassociative = {{work}
{\textbf{Work}}
{
  \textbf{ID} \\
  \hline
  title \\
  is part of \\
  is modeled on \\
}
}};


\pic [right = 12em of work] { entityassociative = {{text}
{\textbf{Text}}
{
  \textbf{ID} \\
  \hline
  is expression of \\
  is modeled on \\
  title \\
  person creator \\
  person translator \\
  creation date \\
  creation date text \\
  creation date cite \\
  matter \\
  regional genre\\
  language\\
  form\\
  poetic meter\\
  stanza length \\
  verse length \\
  rhyme perfection\\
}
}};

\draw[one-omany] (cycle.east) -- node[label, above]{Work is part of} (work.west);
\draw[one-omany] (cycle.east) -- node[label, below]{1 and only 1 Cycle} (work.west);

\draw[one-many] (work.east) -- node[label, above]{Work has 1 or more} (text.west);
\draw[one-many] (work.east) -- node[label, below]{Text expressions} (text.west);

\end{tikzpicture}
    \end{center}
\label{fig:WorkER}
\caption{\textit{Work} entity relationships.}
\end{figure}

%%%%%%%%%%%%%%%%%%%%%%%%%

\subsection{Text}

Definition: Formulation of a \textit{Work} in human language, whose literary form and style can be detected and whose creation can be attributed to one or more individuals.

\vspace{1em}
\noindent Attributes:
\begin{itemize}
    \item \texttt{is expression of} (foreign key [\textbf{Work}], req., uniq.): The \textit{Work} that the \textit{Text} articulates.
    \item \texttt{is modeled on} (foreign key, [\textbf{Text}], opt., uniq.): If the \textit{Text} is derived from another \textit{Text}, a reference to the model \textit{Text}.
    \item \texttt{title} (text, req., uniq.): Either the given title of the \textit{Text}, as provided by the creator, or the standardized title most used in scholarship to refer to the \textit{Text}.
    \item \texttt{is written by} (foreign key [\textbf{Person}], opt., repeat.): The individual accredited with composing the \textit{Text}.
    \item \texttt{is translated by} (foreign key [\textbf{Person}], opt., repeat.): When the \textit{Text} is a translation of another \textit{Text}, the individual accredited with creating the translation.
    \item \texttt{creation date} (list[date], opt., uniq.): A list of two or one dates; the first date is either the earliest or the only date associated with the \textit{Text's} creation, and, in the case of a range, the second date is the latest date associated with the creation.
    \item \texttt{creation date text} (text, opt., uniq.): The date associated with the \textit{Text's} creation as it is written in a scholarly source.
    \item \texttt{creation date cite} (text, opt., uniq.): A citation of the source that provided the date of creation.
    \item \texttt{matter} (terms, opt., repeat.): The matter treated in the \textit{Text}, as defined by Jean Bodel.
    \begin{itemize}
        \item \texttt{Britain}: Matter of Britain, which includes stories about Tristan and King Arthur.
        \item \texttt{France}: Matter of France, which includes stories about Charlemagne.
        \item \texttt{Rome}: Matter of Rome, which includes stories about Troy, Rome, Alexander, and antiquity.
    \end{itemize}
    \item \texttt{regional genre} (terms, req., uniq.): Literary genre attributed to the \textit{Text}.
    \begin{multicols}{2}
        \begin{itemize}
            \item Relevant to French tradition
                \begin{itemize}
                    \item \texttt{chanson de geste}
                    \item \texttt{roman}
                    \item \brackettext{needs development}
                \end{itemize}
            \item Relevant to Iberian tradition
            \begin{itemize}
                \item \texttt{romancero}
                \item \texttt{novela}
                \item \brackettext{needs development}
            \end{itemize}
            \item Relevant to Italian tradition
            \begin{itemize}
                \item \href{https://www.oxfordbibliographies.com/display/document/obo-9780195396584/obo-9780195396584-0199.xml}{\texttt{cantare}}
                \item \texttt{poema cavalleresche}
                \item \brackettext{needs development}
            \end{itemize}
            \item Relevant to Islandic tradition
                \begin{itemize}
                    \item \texttt{fornaldarsögur}
                    \item \texttt{riddarasögur}
                    \item \texttt{rímur}
                    \item \texttt{fornaldarrímur}
                    \item \texttt{riddararímur}
                    \item \brackettext{needs development}
                \end{itemize}
            \item Relevant to Middle Dutch tradition
                \begin{itemize}
                    \item \texttt{ridderepiek}
                    \item \texttt{ridderroman}
                    \item \texttt{rijmkronieken}
                    \item \brackettext{needs development}
                \end{itemize}
            \item Relevant to Middle English, Middle Irish, Middle Welsh traditions
            \begin{itemize}
                \item \texttt{romance}
                \item \brackettext{needs development}
            \end{itemize}
            \item Relevant to Middle High German tradition
            \begin{itemize}
                \item \brackettext{needs development}
            \end{itemize}
        \end{itemize}
    \end{multicols}
    \item \texttt{language} (terms, req., uniq.): ISO code of the primary language through which the \textit{Text} expresses the \textit{Work}.
    \begin{multicols}{2}
        \begin{itemize}
            \item \texttt{cat}: Catalan
            \item \texttt{dum}: Middle Dutch
            \item \texttt{enm}: Middle English (1100-1500)
            \item \texttt{frm}: Middle French (ca. 1400-1600)
            \item \texttt{fro}: Old French (842-ca. 1400)
            \item \href{https://www.wikidata.org/wiki/Q54879035}{\texttt{fro\_ITA}}: Franco-Italian
            \item \texttt{fro\_PRO}: Franco-Occitan
            \item \texttt{ghg}: Hiberno-Scottish Gaelic, Early Modern Irish
            \item \texttt{glg}: Galician
            \item \href{https://www.wikidata.org/wiki/Q1072111}{\texttt{glg\_POR}}: Galician-Portugese
            \item \texttt{gmh}: Middle High German (ca. 1050-1500)
            \item \texttt{gml}: Middle Low German
            \item \texttt{isl}: Islandic
            \item \texttt{ita}: Italian
            \item \texttt{mga}: Middle Irish (900-1200)
            \item \texttt{non}: Old Norse
            \item \href{https://www.wikidata.org/wiki/Q12330003}{\texttt{non\_DAN}}: Old East Norse, Old Danish (800-1100)
            \item \href{https://www.wikidata.org/wiki/Q2417210}{\texttt{non\_SWE}}: Old Swedish (800-1500)
            \item \texttt{oco}: Old Cornish
            \item \texttt{por}: Portugese
            \item \texttt{pro}: Old Occitan, Old Provençal (to 1500)
            \item \texttt{spa}: Spanish or Castilian
            \item \texttt{wlm}: Middle Welsh
            \item \href{https://data.biblissima.fr/w/Item:Q286307}{\texttt{xno}}: Anglo-French, Anglo-Norman
        \end{itemize}
    \end{multicols}
    \item \texttt{form} (terms, req., uniq.): Whether the \textit{Text} is formed in prose, verse, or a mix of both.
    \begin{multicols}{4}
        \begin{itemize}
            \item \texttt{prose}
            \item \texttt{verse}
            \item \texttt{mixed}
        \end{itemize}
    \end{multicols}
    \item \texttt{poetic meter} (terms, opt., uniq.): If the \textit{Text} is in verse, the type of poetic meter.
        \begin{itemize}
            \item \texttt{French alexandrine}: line consisting of 2 half-lines each of 6 syllables, total of 12 syllables.
            \item \texttt{dodecasyllabe}: line consisting of 12 syllables.
            \item \texttt{decasyllabe}: line consisting of 10 syllables.
            \item \texttt{octosyllabe}: line consisting of 8 syllables.
            \item \texttt{hexasyllabe}: line consisting of 6 syllables.
            \item \texttt{pentasyllabe}: line consisting of 5 syllables.
        \end{itemize}
    \item \texttt{rhyme type} (terms, opt., repeat.): If the \textit{Text} is in verse, the types of rhyme used.
        \begin{itemize}
            \item \texttt{alliteration}: Rhyme is allowed between words that either start with the same consonant sound or with the same vowel, such as ``\textit{With floures fele, fair under fete}'' in Middle English.\footcite[][396]{Davis2002}
            \item \texttt{assonance}: Rhyme is allowed between words that have a repeated vowel sound, such as ``a'' in ``\textit{Vio puertas abiertas e uços sin cañados / alcandaras vazias sin pielles e sin mantos}'' in Castilian.\footcite[][364]{Gornall1995}
            \item \texttt{end-rhyme}: Rhyme is allowed between words that have an ending that sounds the same, such as ``\textit{ihesu guz son ihesu goþe / bløt mit hiærta mæþ þino bloþe}'' in Old Norse.\footcite[][423]{Layher2008}
            \item \texttt{generic}: ``[R]hyme is allowed between any one member of a phonetic group and is itself or any other member of the same group,'' such as `b,' `g,' `d' in Old Irish.\footcite[][822]{McKie1997}
        \end{itemize}
    \item \texttt{strophe count} (integer, opt., uniq.): If the \textit{Text} is in verse and strophic, the number of strophes.
    \item \texttt{strophe length} (integer, opt., uniq.): If the \textit{Text} is in verse and strophic, the number of lines in each strophe.
    \item \texttt{verse length} (integer, opt., uniq.): If the \textit{Text} is in verse, the number of verses a complete version (\textit{Witness}) of it should have.
\end{itemize}

\begin{figure}[ht]
    \begin{center}
        \tikzstyle{s} = [rectangle, rounded corners, minimum width=2cm, text width=3cm, minimum height=1cm, text centered, draw=black]
\tikzstyle{arrow} = [thick,->,>=stealth]
\begin{tikzpicture}[-,shorten >=1pt,auto,node distance=1.5cm,semithick]
\tikzstyle{every state}=[fill=red,draw=none,text=white]

\pic [] { entityassociative = {{text}
{\textbf{Text}}
{
  \textbf{ID} \\
  \hline
  is expression of \\
  is modeled on \\
  title \\
  is written by \\
  is translated by \\
  creation date \\
  creation date text \\
  creation date cite \\
  matter \\
  regional genre\\
  language\\
  form\\
  poetic meter\\
  rhyme type \\
  strophe count \\
  strophe length \\
  verse length \\
}
}};

\pic [left = 12em of text] { entityassociative = {{work}
{\textbf{Work}}
{
  \textbf{ID} \\
  \hline
  title \\
  is part of \\
}
}};

\pic [right = 12em of text] { entityassociative = {{witness}
{\textbf{Witness}}
{
  \textbf{ID} \\
  \hline
  is manifestation of\\
  is preceded by fragm.\\
  is preceded by wit.\\
  is visible on\\
  was contained in\\
  siglum \\
  status \\
  TEI document\\
  IIIF manifest\\
  scribe\\
  scripta\\
  creation date\\
  creation date text\\
  note\\
}
}};

\pic [below = 7em of text] { entityassociative = {{person}
{\textbf{Person}}
{
  \textbf{ID} \\
  \hline
  name \\
  note
}
}};

\pic [above = 7em of text] { entityassociative = {{reference}
{\textbf{Reference}}
{
  entity type\\
  entity ID \\
  unique identifier \\
  identifier source \\
  permalink\\
  citation \\
}
}};

\draw[omany-omany] (reference) -- node[label, pos=0.5, right]{Text has 0, 1, or many References} (text);

\draw[one-omany] (text) -- node[label, above, yshift=0.10cm]{Text is manifest in 0, 1, or} (witness);
\draw[one-omany] (text) -- node[label, below, yshift=-0.10cm]{many Witnesses} (witness);

\draw[one-many] (text) -- node[label, above, yshift=0.10cm]{Text is expression of} (work);
\draw[one-many] (text) -- node[label, below, yshift=-0.10cm]{1 and only 1 Work} (work);

\draw[omany-omany] (text) -- node[label, pos=0.40, right]{Text is written by 0, 1, or many Persons} (person);
\draw[omany-omany] (text) -- node[label, pos=0.60, right]{Text is translated by 0, 1, or many Persons} (person);

\end{tikzpicture}
    \end{center}
\label{fig:TextER}
\caption{\textit{Text} entity relationships.}
\end{figure}

%%%%%%%%%%%%%%%%%%%%%%%%%

\subsection{Person}

Definition: An individual bearing some responsibility for a \textit{Text} in the data set, either as a creator/author or as a translator, in the case of a \textit{Text} that is the translation of a model \textit{Text} in another language.

\vspace{1em}
\noindent Attributes:
\begin{itemize}
    \item \texttt{name} (text, req., uniq.): The recieved name of the individual.
    \item \texttt{note} (text, opt., uniq.): Optional notes to help identify the individual.
\end{itemize}

\begin{figure}[ht]
    \begin{center}
        \tikzstyle{s} = [rectangle, rounded corners, minimum width=2cm, text width=3cm, minimum height=1cm, text centered, draw=black]
\tikzstyle{arrow} = [thick,->,>=stealth]
\begin{tikzpicture}[-,shorten >=1pt,auto,node distance=1.5cm,semithick]
\tikzstyle{every state}=[fill=red,draw=none,text=white]

\pic [] { entityassociative = {{text}
{\textbf{Text}}
{
  \textbf{ID} \\
  \hline
  is expression of \\
  is modeled on \\
  title \\
  person creator \\
  person translator \\
  creation date \\
  creation date text \\
  creation date cite \\
  matter \\
  regional genre\\
  language\\
  form\\
  poetic meter\\
  stanza length \\
  verse length \\
  rhyme perfection\\
}
}};


\pic [right = 12em of text] { entityassociative = {{person}
{\textbf{Person}}
{
  \textbf{ID} \\
  \hline
  name \\
  birth year \\
  death year \\
  location
}
}};

\draw[omany-omany] (text) -- node[label, above]{Person wrote/translated} (person);
\draw[omany-omany] (text) -- node[label, below]{0 or many Texts} (person);

\end{tikzpicture}
    \end{center}
\label{fig:PersonER}
\caption{\textit{Person} entity relationships.}
\end{figure}

%%%%%%%%%%%%%%%%%%%%%%%%%

\subsection{Witness}

Definition: An extant manifestation of a \textit{Text} in a defined sequence of characters, which have been inscribed on a physical document.

\vspace{1em}
\noindent Attributes:
\begin{itemize}
    \item \texttt{is manifestation of} (foreign key [\textbf{Text}], req., uniq.): The \textit{Text} whose linguistic content the \textit{Witness} manifests in writing.
    \item \texttt{is preceded by} (foreign key [\textbf{Witness}], opt., uniq.): If the \textit{Witness} is a fragment, reference to another \textit{Witness} fragment, with which the \textit{Witness} is thought to have been produced in an original document, and which presents an earlier part of the \textit{Text} than its own part.
    \item \texttt{is visible on} (foreign key [\textbf{Pages}], opt., repeat.): 
    \item \texttt{was contained in} (foreign key [\textbf{Attested Document}], opt., uniq.): 
    \item \texttt{siglum} (text, opt., repeat.): Identifier used by scholars to indicate the extant \textit{Witness} within the \textit{Text's} tradition.
    \item \texttt{status} (terms, req., uniq.): Status of the extant \textit{Witness}.
    \begin{itemize}
        \item \texttt{complete}: All pages containing the main textual content of the \textit{Witness}, excluding dedications and decorations, have survived.
        \item \texttt{mutilated}: The \textit{Witness} presents parts of its \textit{Text} on a set of surviving pages ($ \geq 3 $ pages).
        \item \texttt{fragment}: Only a few pages of the \textit{Witness} survive ($ \leq 2 $ pages). 
        \item \texttt{citation}: The \textit{Witness} testifies to the existence of a \textit{Text}, through citation, but does not present all of the latter's linguistic content.
    \end{itemize}
    \item \texttt{TEI document} (text, opt., uniq.): Reference to a TEI-XML file representing the text content of the \textit{Witness}.
    \item \texttt{IIIF manifest} (text, opt., uniq.): Reference to a IIIF manifest JSON file representing the digitized images of the \textit{Witness's} \textit{Pages}.
    \item \texttt{scribe} (text, opt., uniq.): When known, identifying information about the scribe alleged to have written the text version.
    \item \texttt{scripta} (terms, opt., uniq.): When known, the name of a regional writing style, similar to the written version of a spoken dialect, as defined by Louis Remacle.\footcite[``Je désigne la langue vulgaire écrite au moyen âge par le néologisme \textit{scripta}. L'expression `la scripta' est synonyme de l'allemand `die Schriftsprache.'''][24]{Remacle1948}
    \begin{multicols}{2}
        \begin{itemize}
            \item \texttt{bourg.}: Bourguignon
            \item \texttt{champ. mérid.}: Champenois méridional
            \item \texttt{champ. setp.}: Champenois septentrional
            \item \texttt{frc.}: francien (Paris)
            \item \texttt{lorr.}: Lorrain
            \item \texttt{lorr. mérid.}: Lorrain méridional
            \item \texttt{lorr. sept.}: Lorrain septentrional (Metz)
            \item \texttt{norm.}: Normand
            \item \texttt{orl.}: Orléanais
            \item \texttt{ouest}: Ouest
            \item \texttt{pic.}: Picard
            \item \texttt{pic. sept.}: Picard septentrional (Flandres)
            \item \texttt{pic. mérid.}: Picard méridional (Soissonnais)
            \item \texttt{wall.}: Wallon
            \item \texttt{liég.}: Liégeois
        \end{itemize}
    \end{multicols}
    \item \texttt{creation date} (list[date], opt., uniq.): A list of two or one dates; the first date is either the earliest or the only date associated with the \textit{Witness's} production, and, in the case of a range, the second date is the latest date associated with the production.
    \item \texttt{creation date text} (text, opt., uniq.): The date associated with the \textit{Witness's} production as it is written in a catalogue or other scholarly source.
    \item \texttt{note} (text, opt., uniq.): Notes about the \textit{Witness} that will not be standardized and used in computational methods but may serve to protect the integrity of the record.
\end{itemize}

\begin{figure}[ht]
    \begin{center}
        \tikzstyle{s} = [rectangle, rounded corners, minimum width=2cm, text width=3cm, minimum height=1cm, text centered, draw=black]
\tikzstyle{arrow} = [thick,->,>=stealth]
\begin{tikzpicture}[-,shorten >=1pt,auto,node distance=1.5cm,semithick]
\tikzstyle{every state}=[fill=red,draw=none,text=white]

\pic [] { entityassociative = {{witness}
{\textbf{Witness}}
{
  \textbf{ID} \\
  \hline
  is manifestation of\\
  is preceded by\\
  is visible on\\
  was contained in\\
  siglum \\
  status \\
  TEI document\\
  IIIF manifest\\
  scribe\\
  scripta\\
  creation date\\
  creation date text\\
  note\\
}
}};

\pic [left = 12em of witness] { entityassociative = {{text}
{\textbf{Text}}
{
  \textbf{ID} \\
  \hline
  is expression of \\
  is modeled on \\
  title \\
  is written by \\
  is translated by \\
  creation date \\
  creation date text \\
  creation date cite \\
  matter \\
  regional genre\\
  language\\
  form\\
  poetic meter\\
  rhyme type \\
  strophe count \\
  strophe length \\
  verse length \\
}
}};

\pic [right = 12em of witness] { entityassociative = {{pages}
{\textbf{Pages}}
  {
    \textbf{ID} \\
    \hline
    is contained in \\
    archival item part \\
    witness part no \\
    folio start \\
    folio end \\
    image start \\
    image end \\
  }
}};

\pic [above = 7em of witness] { entityassociative = {{reference}
{\textbf{Reference}}
{
  entity type\\
  entity ID \\
  unique identifier \\
  identifier source \\
  permalink\\
  citation \\
}
}};

\pic [below = 5em of witness] { entityassociative = {{document}
{\textbf{Attested Document}}
{
  \textbf{ID} \\
  \hline
  name\\
  creation date\\
  creation date text\\
}
}};

\draw[omany-omany] (reference) -- node[label, pos=0.5, right]{Witness has 0, 1, or many References} (witness);

\draw[one-omany] (text) -- node[label,above]{Witness is manifestation} (witness);
\draw[one-omany] (text) -- node[label,below]{of 1 and only 1 Text} (witness);

\draw[one-many] (witness) --node[label, above]{Witness is visible on} (pages);
\draw[one-many] (witness) --node[label, below]{1 or many Pages} (pages);

\draw[omany-omany] (document) -- node[label, pos=0.5, right]{Witness was contained in 0, 1, or many Attested Documents} (witness);

\end{tikzpicture}
    \end{center}
\label{fig:WitnessER}
\caption{\textit{Witness} entity relationships.}
\end{figure}

%%%%%%%%%%%%%%%%%%%%%%%%%

\subsection{Attested Document}

Definition: Hypothetical text object, conceived of and produced as one unit, in which \textit{Witness} were transmitted; it no longer exists today.

\vspace{1em}
\noindent Attributes:

\begin{itemize}
    \item \texttt{name} (text, req., uniq.): Name given to the \textit{Attested Witness} used to help identify it.
    \item \texttt{creation date} (list[date], opt., uniq.): A list of two or one dates; the first date is either the earliest or the only date associated with the \textit{Attested Witness's} production, and, in the case of a range, the second date is the latest date associated with the production.
    \item \texttt{creation date text} (text, opt., uniq.): The date associated with the \textit{Attested Witness's} production as it is written in a catalogue or other scholarly source.
\end{itemize}

\begin{figure}[ht]
    \begin{center}
        \tikzstyle{s} = [rectangle, rounded corners, minimum width=2cm, text width=3cm, minimum height=1cm, text centered, draw=black]
\tikzstyle{arrow} = [thick,->,>=stealth]
\begin{tikzpicture}[-,shorten >=1pt,auto,node distance=1.5cm,semithick]
\tikzstyle{every state}=[fill=red,draw=none,text=white]

\pic [] { entityassociative = {{witness}
{\textbf{Witness}}
{
  \textbf{ID} \\
  \hline
  is manifestation of\\
  is preceded by\\
  is visible on\\
  was contained in\\
  siglum \\
  status \\
  TEI document\\
  IIIF manifest\\
  scribe\\
  scripta\\
  creation date\\
  creation date text\\
  note\\
}
}};

\pic [right = 15em of witness] { entityassociative = {{doc}
{\textbf{Attested Document}}
{
  \textbf{ID} \\
  \hline
  name\\
  creation date\\
  creation date text\\
}
}};

\draw[one-omany] (witness) --node[label, above]{Attested Document contained} (doc);
\draw[one-omany] (witness) --node[label, below]{1 or more Witnesses} (doc);



\end{tikzpicture}
    \end{center}
\label{fig:AttestedDocumentER}
\caption{\textit{Attested Document} entity relationships.}
\end{figure}

%%%%%%%%%%%%%%%%%%%%%%%%%

\subsection{Pages}

Definition: An uninterrupted series of leafs or pages in an \textit{Archival Item} on which is inscribed the \textit{Witness} of a \textit{Text}.

\vspace{1em}
\noindent Attributes:
\begin{itemize}
    \item \texttt{is contained in} (foreign key [\textbf{Archival Item}], req., uniq.): The physical archival item in which the pages are currently bound or otherwise stored.
    \item \texttt{archival item part} (text, opt., uniq.): If the item's archival description and/or catalogue entry has subdivided the item into parts (i.e. Part II), the part containing the \textit{Pages}; this information is used to enrich bibliographic references.
    \item \texttt{witness part} (integer, opt., uniq.): If the \textit{Witness} is composed of discrete parts (i.e. multiple volumes), the \textit{Pages}' order in the sequence; this information is used to correctly order multiple \textit{Pages} entities descending from a \textit{Witness}.
    \item \texttt{folio start} (text, req., uniq.): The folio number inscribed on the first page.
    \item \texttt{folio end} (text, req., uniq.): The folio number inscribed on the last page.
    \item \texttt{image start} (integer, opt., uniq.): If the pages are digitized, the number of the first digital image.
    \item \texttt{image end} (integer, opt., uniq.): If the pages are digitized, the number of the last digital image.
\end{itemize}

\begin{figure}[ht]
    \begin{center}
        \tikzstyle{s} = [rectangle, rounded corners, minimum width=2cm, text width=3cm, minimum height=1cm, text centered, draw=black]
\tikzstyle{arrow} = [thick,->,>=stealth]
\begin{tikzpicture}[-,shorten >=1pt,auto,node distance=1.5cm,semithick]
\tikzstyle{every state}=[fill=red,draw=none,text=white]

\pic [] { entityassociative = {{pages}
{\textbf{Pages}}
{
    \textbf{ID} \\ \hline
    is contained in \\
    archival item part \\
    witness part \\
    folio start \\
    folio end \\
    image start \\
    image end \\
}
}};

\pic [left = 12em of pages] { entityassociative = {{witness}
{\textbf{Witness}}
{
  \textbf{ID} \\
  \hline
  is manifestation of\\
  is preceded by\\
  is visible on\\
  was contained in\\
  siglum \\
  status \\
  TEI document\\
  IIIF manifest\\
  scribe\\
  scripta\\
  creation date\\
  creation date text\\
  note\\
}
}};

\pic [right = 12em of pages] { entityassociative = {{item}
{\textbf{Archival Item}}
{
  \textbf{ID} \\
  \hline
  is conserved in\\
  collection\\
  shelfmark\\
  belatedly compiled\\
  provenance\\
  digitisation URL\\
  IIIF manifest\\
}
}};

\draw[one-omany] (witness) --node[label, above]{Pages makes visible} (pages);
\draw[one-omany] (witness) --node[label, below]{1 and only 1 Witness} (pages);

\draw[omany-one] (pages) --node[label, above]{Pages are contained in 1} (item);
\draw[omany-one] (pages) --node[label, below]{and only 1 Archival Item} (item);

\end{tikzpicture}
    \end{center}
\label{fig:PagesER}
\caption{\textit{Pages} entity relationships.}
\end{figure}

%%%%%%%%%%%%%%%%%%%%%%%%%

\subsection{Archival Item}

Definition: An extant physical document conserved in an archival institution.

\vspace{1em}
\noindent Attributes:

\begin{itemize}
    \item \texttt{is conserved in} (foreign key [\textbf{Repository}], req., uniq.): The archival institution that currently holds the item.
    \item \texttt{collection} (text, opt., uniq.): The name of a collection in the archival institution.
    \item \texttt{shelfmark} (text, req., uniq.): The shelfmark of the item in the archival institution.
    \item \texttt{belatedly compiled} (boolean, req., uniq.): Whether the item is a collection of manuscripts that pre-date the item's production and were extracted from their original contexts, belatedly compiled together (i.e 19th-century collection of Medieval fragments).
    \item \texttt{provenance} (text, opt., uniq.): Notes about the provenance of the \textit{Archival Item}.
    \item \texttt{digitisation URL} (text, opt., uniq.): If digitized, the URL to the digital document.
    \item \texttt{IIIF manifest} (text, opt., uniq.): If a IIIF manifest is available of the digitized document, a link to the manifest's JSON.
\end{itemize}

\begin{figure}[ht]
    \begin{center}
        \tikzstyle{s} = [rectangle, rounded corners, minimum width=2cm, text width=3cm, minimum height=1cm, text centered, draw=black]
\tikzstyle{arrow} = [thick,->,>=stealth]
\begin{tikzpicture}[-,shorten >=1pt,auto,node distance=1.5cm,semithick]
\tikzstyle{every state}=[fill=red,draw=none,text=white]

\pic [] { entityassociative = {{pages}
{\textbf{Pages}}
{
    \textbf{ID} \\ \hline
    is contained in \\
    archival item part \\
    witness part \\
    folio start \\
    folio end \\
    image start \\
    image end \\
}
}};

\pic [right = 12em of pages] { entityassociative = {{item}
{\textbf{Archival Item}}
{
  \textbf{ID} \\
  \hline
  is conserved in\\
  collection\\
  shelfmark\\
  belatedly compiled\\
  provenance\\
  digitisation URL\\
  IIIF manifest\\
}
}};

\pic [right = 12em of item] { entityassociative = {{repository}
{\textbf{Repository}}
{
  \textbf{ID} \\
  \hline
  institutional name\\
  street address\\
  website\\
  AEP reference\\
  country\\
  WikiData ID\\
  GeoNames ID\\
  GeoShape Map\\
}
}};


\draw[omany-one] (pages) --node[label, above]{Archival Item contains} (item);
\draw[omany-one] (pages) --node[label, below]{0, 1, or many Pages} (item);

\draw[omany-one] (item) --node[label, above]{Archival item is conserved} (repository);
\draw[omany-one] (item) --node[label, below]{1 and only 1 Repository} (repository);

\end{tikzpicture}
    \end{center}
\label{fig:ArchivalER}
\caption{\textit{Archival Item} entity relationships.}
\end{figure}

%%%%%%%%%%%%%%%%%%%%%%%%%

\subsection{Repository}

Definition: An archival institution.

\vspace{1em}
\noindent Attributes:

\begin{itemize}
    \item \texttt{institutional name} (text, req., uniq.): The full, official name of the archival institution in the original language.
    \item \texttt{street address} (text, opt., uniq.): The institution's principal address, which could be used to localize it.
    \item \texttt{website} (text, opt., uniq.): A link to the homepage of the institution's website.
    \item \texttt{AEP reference} \brackettext{\texttt{Archives Portal Europe reference}} (text, opt., uniq.): URL to the archival institution's information on the Archives Portal Europe website.
    \item \texttt{country} (terms, opt., uniq.): Country in which the institution is located.
    \item \texttt{WikiData ID} (text, opt., uniq.): WikiData ID of the country in which the institution is located.
    \item \texttt{GeoNames ID} (integer, opt., uniq.): GeoNames ID of the country in which the institution is located.
    \item \texttt{GeoShape Map} (text, opt., uniq.): Link to a GeoShape map on Wikimedia.
\end{itemize}

\begin{figure}[ht]
    \begin{center}
        \tikzstyle{s} = [rectangle, rounded corners, minimum width=2cm, text width=3cm, minimum height=1cm, text centered, draw=black]
\tikzstyle{arrow} = [thick,->,>=stealth]
\begin{tikzpicture}[-,shorten >=1pt,auto,node distance=1.5cm,semithick]
\tikzstyle{every state}=[fill=red,draw=none,text=white]

\pic [] { entityassociative = {{item}
{\textbf{Archival Item}}
{
  \textbf{ID} \\
  \hline
  is conserved in\\
  collection\\
  shelfmark\\
  belatedly compiled\\
  provenance\\
  digitisation URL\\
  IIIF manifest\\
}
}};

\pic [right = 15em of item] { entityassociative = {{repository}
{\textbf{Repository}}
{
  \textbf{ID} \\
  \hline
  institutional name\\
  street address\\
  website\\
  AEP reference\\
  country\\
  WikiData ID\\
  GeoNames ID\\
  GeoShape Map\\
}
}};

\draw[omany-one] (item) --node[label, above]{Repository conserves 0, 1,} (repository);
\draw[omany-one] (item) --node[label, below]{or many Archival Items} (repository);

\end{tikzpicture}
    \end{center}
\label{fig:RepositoryER}
\caption{\textit{Repository} entity relationships.}
\end{figure}

%%%%%%%%%%%%%%%%%%%%%%%%%